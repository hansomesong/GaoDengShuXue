% -*- coding: utf-8 -*-
% !TEX program = xelatex

%\documentclass[14pt]{article}
%\usepackage[notheorems]{beamerarticle}

\documentclass[14pt,notheorems,leqno,xcolor={rgb}]{beamer} % ignorenonframetext

% -*- coding: utf-8 -*-
% ----------------------------------------------------------------------------
% Author:  Jianrui Lyu <tolvjr@163.com>
% Website: https://github.com/lvjr/theme
% License: Creative Commons Attribution-ShareAlike 4.0 International License
% ----------------------------------------------------------------------------

\ProvidesPackage{beamerthemeriemann}[2018/06/05 v0.6 Beamer Theme Riemann]

\makeatletter

% compatible with old versions of beamer
\providecommand{\beamer@endinputifotherversion}[1]{}

\RequirePackage{tikz,etoolbox,adjustbox}
\usetikzlibrary{shapes.multipart}

\mode<presentation>

\setbeamersize{text margin left=8mm,text margin right=8mm}

%% ----------------- background canvas and background ----------------

\newif\ifbackgroundmarkleft
\newif\ifbackgroundmarkright

\newcommand{\insertbackgroundmark}{
  \ifbackgroundmarkleft
    \foreach \x in {1,2,3,4,5} \draw[very thick,markcolor] (0,\x*\paperheight/6) -- +(1.2mm,0);
  \fi
  \ifbackgroundmarkright
    \foreach \x in {1,2,3,4,5} \draw[very thick,markcolor] (\paperwidth,\x*\paperheight/6) -- +(-1.2mm,0);
  \fi
}

\defbeamertemplate{background}{line}{%
  \begin{tikzpicture}
    \useasboundingbox (0,0) rectangle (\paperwidth,\paperheight);
    \draw[xstep=\paperwidth,ystep=1mm,color=tcolor] (0,0) grid (\paperwidth,\paperheight);
    \insertbackgroundmark
  \end{tikzpicture}%
}

\defbeamertemplate{background}{linear}{%
  \begin{tikzpicture}
    \useasboundingbox (0,0) rectangle (\paperwidth,\paperheight);
    \draw[pattern=horizontal lines, pattern color=tcolor]
      (0,0) rectangle (\paperwidth,\paperheight);
    \insertbackgroundmark
  \end{tikzpicture}%
}

\defbeamertemplate{background}{lattice}[1][1mm]{%
  \begin{tikzpicture}
    \useasboundingbox (0,0) rectangle (\paperwidth,\paperheight);
    \draw[step=#1,color=tcolor,semithick] (0,0) grid (\paperwidth,\paperheight);
    \insertbackgroundmark
  \end{tikzpicture}%
}

\defbeamertemplate{background}{empty}{
  \begin{tikzpicture}
    \useasboundingbox (0,0) rectangle (\paperwidth,\paperheight);
    \insertbackgroundmark
  \end{tikzpicture}%
}

%% -------------------------- title page -----------------------------

% add \occasion command
\newcommand{\occasion}[1]{\def\insertoccasion{#1}}
\occasion{}

\defbeamertemplate{title page}{banner}{%
  \nointerlineskip
  \begin{adjustbox}{width=\paperwidth,center}%
    \usebeamertemplate{title page content}%
  \end{adjustbox}%
}

% need "text badly ragged" option for correct space skips
% see http://tex.stackexchange.com/a/132748/8956
\defbeamertemplate{title page content}{hexagon}{%
  \begin{tikzpicture}
  \useasboundingbox (0,0) rectangle (\paperwidth,\paperheight);
  \path[draw=dcolor,fill=fcolor,opacity=0.8]
      (0,0) rectangle (\paperwidth,\paperheight);
  \node[text width=0.86\paperwidth,text badly ragged,inner ysep=1.5cm] (main) at (0.5\paperwidth,0.55\paperheight) {%
    \begin{minipage}[c]{0.86\paperwidth}
      \centering
      \usebeamerfont{title}\usebeamercolor[fg]{title}\inserttitle
      \ifx\insertsubtitle\@empty\else
        \\[5pt]\usebeamerfont{subtitle}\usebeamercolor[fg]{subtitle}
        \insertsubtitle
      \fi
    \end{minipage}
  };
  \node[rectangle,inner sep=0pt,minimum size=3mm,fill=dcolor,right] (a) at (0,0.55\paperheight) {};
  \node[rectangle,inner sep=0pt,minimum size=3mm,fill=dcolor,left] (b) at (\paperwidth,0.55\paperheight) {};
  \ifx\insertoccasion\@empty
      \draw[thick,dcolor] (a.north east) -- (main.north west)
                   -- (main.north east) -- (b.north west);
  \else
      \node[text badly ragged] (occasion) at (main.north west -| 0.5\paperwidth,\paperheight) {
          \usebeamerfont{occasion}\usebeamercolor[fg]{occasion}\insertoccasion
      };
      \draw[thick,dcolor] (a.north east) -- (main.north west) -- (occasion.west)
                          (b.north west) -- (main.north east) -- (occasion.east);
  \fi
  \node[text badly ragged] (date) at (main.south west -| 0.5\paperwidth,0) {
      \usebeamerfont{date}\usebeamercolor[fg]{date}\insertdate
  };
  \draw[thick,dcolor] (a.south east) -- (main.south west) -- (date.west)
                      (b.south west) -- (main.south east) -- (date.east);
  \node[below=4mm,text width=0.9\paperwidth,inner xsep=0.05\paperwidth,
        text badly ragged,fill=white] at (date.south) {%
      \begin{minipage}[c]{0.9\paperwidth}
          \centering
          \textcolor{brown75}{$\blacksquare$}\hspace{0.2em}%
          \usebeamerfont{institute}\usebeamercolor[fg]{institute}\insertinstitute
          \hspace{0.4em}\textcolor{brown75}{$\blacksquare$}\hspace{0.2em}%
          \usebeamerfont{author}\usebeamercolor[fg]{author}\insertauthor
      \end{minipage}
  };
  \end{tikzpicture}
}

\defbeamertemplate{title page content}{rectangle}{%
  \begin{tikzpicture}
  \useasboundingbox (0,0) rectangle (\paperwidth,\paperheight);
  \path[draw=dcolor,fill=fcolor,opacity=0.8]
      (0,0.25\paperheight) rectangle (\paperwidth,0.85\paperheight);
  \path[draw=dcolor,very thick]
    %%(0.0075\paperwidth,0.26\paperheight) rectangle (0.9925\paperwidth,0.84\paperheight);
      (0.0375\paperwidth,0.26\paperheight) -- (0.9625\paperwidth,0.26\paperheight)
         -- ++(0,0.02\paperheight) -- ++(0.03\paperwidth,0)
         -- ++(0,-0.02\paperheight) -- ++(-0.015\paperwidth,0)
         -- ++(0,0.04\paperheight) -- ++(0.015\paperwidth,0)
      -- (0.9925\paperwidth,0.8\paperheight)
         -- ++(-0.015\paperwidth,0) -- ++(0,0.04\paperheight)
         -- ++(0.015\paperwidth,0) -- ++(0,-0.02\paperheight)
         -- ++(-0.03\paperwidth,0) -- ++(0,0.02\paperheight)
      -- (0.0375\paperwidth,0.84\paperheight)
         -- ++(0,-0.02\paperheight) -- ++(-0.03\paperwidth,0)
         -- ++(0,0.02\paperheight) -- ++(0.015\paperwidth,0)
         -- ++(0,-0.04\paperheight) -- ++(-0.015\paperwidth,0)
      -- (0.0075\paperwidth,0.3\paperheight)
         -- ++(0.015\paperwidth,0) -- ++(0,-0.04\paperheight)
         -- ++(-0.015\paperwidth,0) -- ++(0,0.02\paperheight)
         -- ++(0.03\paperwidth,0) -- ++(0,-0.02\paperheight)
      -- cycle;
  \node[text width=0.9\paperwidth,text badly ragged] at (0.5\paperwidth,0.55\paperheight) {%
    \begin{minipage}[c][0.58\paperheight]{0.9\paperwidth}
      \centering
      \usebeamerfont{title}\usebeamercolor[fg]{title}\inserttitle
      \ifx\insertsubtitle\@empty\else
        \\[5pt]\usebeamerfont{subtitle}\usebeamercolor[fg]{subtitle}
        \insertsubtitle
      \fi
    \end{minipage}
  };
  \ifx\insertoccasion\@empty\else
    \node[text badly ragged,below,draw=dcolor,fill=white] at (0.5\paperwidth,0.84\paperheight) {%
      \usebeamerfont{occasion}\usebeamercolor[fg]{occasion}\insertoccasion
    };
  \fi
  \node[text width=0.9\paperwidth,text badly ragged,below] at (0.5\paperwidth,0.25\paperheight) {%
    \begin{minipage}[t][0.25\paperheight]{0.9\paperwidth}
      \centering
      {\color{brown75}$\blacksquare$}
      \usebeamerfont{institute}\usebeamercolor[fg]{institute}\insertinstitute
      \hfill
      {\color{brown75}$\blacksquare$}
      \usebeamerfont{author}\usebeamercolor[fg]{author}\insertauthor
      \hfill
      {\color{brown75}$\blacksquare$}
      \usebeamerfont{date}\usebeamercolor[fg]{date}%
      \the\year-\ifnum\month<10 0\fi\the\month-\ifnum\day<10 0\fi\the\day
    \end{minipage}
  };
  \end{tikzpicture}
}

%% ----------------------- section and subsection --------------------

\newcounter{my@pgf@picture@count}

\def\sectionintocskip{0.5pt plus 0.1fill}
\patchcmd{\beamer@sectionintoc}{\vskip1.5em}{\vskip\sectionintocskip}{}{}

\AtBeginSection[]{%
  \begin{frame}%[plain]
    \sectionpage
  \end{frame}%
}

\defbeamertemplate{section name}{simple}{\insertsectionnumber.}

\defbeamertemplate{section name}{chinese}[1][节]{第\CJKnumber{\insertsectionnumber}#1}

\defbeamertemplate{section page}{single}{%
  \centerline{%
    \usebeamerfont{section name}%
    \usebeamercolor[fg]{section name}%
    \usebeamertemplate{section name}%
    \hspace{0.8em}%
    \usebeamerfont{section title}%
    \usebeamercolor[fg]{section title}%
    \insertsection
  }%
}

\defbeamertemplate{section name in toc}{simple}{%
  Section \ifnum\the\beamer@tempcount<10 0\fi\inserttocsectionnumber
}

\defbeamertemplate{section name in toc}{chinese}[1][节]{%
  第\CJKnumber{\inserttocsectionnumber}#1%
}

\newcounter{my@section@from}
\newcounter{my@section@to}

\defbeamertemplate{show sections in toc}{total}{%
  \setcounter{my@section@from}{1}%
  \setcounter{my@section@to}{50}%
}

% show at most five sections
\defbeamertemplate{show sections in toc}{partial}{%
  \setcounter{my@section@from}{\value{section}}%
  \addtocounter{my@section@from}{-2}%
  \setcounter{my@section@to}{\value{section}}%
  \addtocounter{my@section@to}{2}%
  \ifnum\my@totalsectionnumber>0%
    \ifnum\value{my@section@to}>\my@totalsectionnumber
      \setcounter{my@section@to}{\my@totalsectionnumber}%
      \setcounter{my@section@from}{\value{my@section@to}}%
      \addtocounter{my@section@from}{-4}%
    \fi
  \fi
  \ifnum\value{my@section@from}<1\setcounter{my@section@from}{1}%
    \setcounter{my@section@to}{\value{my@section@from}}%
    \addtocounter{my@section@to}{4}%
  \fi
}

% reset pgfid to get correct result with \tikzmark in second run
\defbeamertemplate{section page}{fill}{%
  \usebeamertemplate{show sections in toc}%
  \setcounter{my@pgf@picture@count}{\the\pgf@picture@serial@count}%
  \tableofcontents[sectionstyle=show/shaded,subsectionstyle=hide,
                   sections={\arabic{my@section@from}-\arabic{my@section@to}}]%
  \global\pgf@picture@serial@count=\value{my@pgf@picture@count}%
  \unskip
}

\defbeamertemplate{section in toc}{fill}{%
  \noindent
  \begin{tikzpicture}
  \node[rectangle split, rectangle split horizontal, rectangle split parts=2,
        rectangle split part fill={sectcolor,bg}, draw=darkgray,
        inner xsep=0pt, inner ysep=5.5pt]
       {
         \nodepart[text width=0.255\textwidth,align=center]{text}
         \usebeamertemplate{section name in toc}
         \nodepart[text width=0.74\textwidth]{second}%
         \hspace{7pt}\inserttocsection
       };
  \end{tikzpicture}%
  \par
}

\AtBeginSubsection{%
  \begin{frame}%[plain]
    \setlength{\parskip}{0pt}%
    \offinterlineskip
    \subsectionpage
  \end{frame}%
}

\defbeamertemplate{subsection name}{simple}{%
  \insertsectionnumber.\insertsubsectionnumber
}

\defbeamertemplate{subsection page}{single}{%
  \centerline{%
    \usebeamerfont{subsection name}%
    \usebeamercolor[fg]{subsection name}%
    \usebeamertemplate{subsection name}%
    \hspace{0.8em}%
    \usebeamerfont{subsection title}%
    \usebeamercolor[fg]{subsection title}%
    \insertsubsection
  }%
}

% reset pgfid to get correct result with \tikzmark in second run
\defbeamertemplate{subsection page}{fill}{%
  \setcounter{my@pgf@picture@count}{\the\pgf@picture@serial@count}%
  \tableofcontents[sectionstyle=show/hide,subsectionstyle=show/shaded/hide]%
  \global\pgf@picture@serial@count=\value{my@pgf@picture@count}%
  \unskip
}

\defbeamertemplate{subsection in toc}{fill}{%
  \noindent
  \begin{tikzpicture}
    \node[rectangle split, rectangle split horizontal, rectangle split parts=2,
          rectangle split part fill={white,bg}, draw=darkgray,
          inner xsep=0pt, inner ysep=5.5pt]
         {
           \nodepart[text width=0.255\textwidth,align=right]{text}
           \inserttocsectionnumber.\inserttocsubsectionnumber\kern7pt%
           \nodepart[text width=0.74\textwidth]{second}%
           \hspace{7pt}\inserttocsubsection
         };
  \end{tikzpicture}%
  \par
}

% chinese sections and subsections
\defbeamertemplate{section and subsection}{chinese}[1][节]{%
  \setbeamertemplate{section name in toc}[chinese][#1]%
  \setbeamertemplate{section name}[chinese][#1]%
}

%% ---------------------- headline and footline ----------------------

\defbeamertemplate{footline left}{author}{%
  \insertshortauthor
}

\defbeamertemplate{footline center}{title}{%
  \insertshorttitle
}

\defbeamertemplate{footline right}{number}{%
  \Acrobatmenu{GoToPage}{\insertframenumber{}/\inserttotalframenumber}%
}
\defbeamertemplate{footline right}{normal}{%
  \hyperlinkframeendprev{$\vartriangle$}
  \Acrobatmenu{GoToPage}{\insertframenumber{}/\inserttotalframenumber}
  \hyperlinkframestartnext{$\triangledown$}%
}

\defbeamertemplate{footline}{simple}{%
  \hbox{%
  \begin{beamercolorbox}[wd=.2\paperwidth,ht=2.25ex,dp=1ex,left]{footline}%
    \usebeamerfont{footline}\kern\beamer@leftmargin
    \usebeamertemplate{footline left}%
  \end{beamercolorbox}%
  \begin{beamercolorbox}[wd=.6\paperwidth,ht=2.25ex,dp=1ex,center]{footline}%
    \usebeamerfont{footline}\usebeamertemplate{footline center}%
  \end{beamercolorbox}%
  \begin{beamercolorbox}[wd=.2\paperwidth,ht=2.25ex,dp=1ex,right]{footline}%
    \usebeamerfont{footline}\usebeamertemplate{footline right}%
    \kern\beamer@rightmargin
  \end{beamercolorbox}%
  }%
}

\defbeamertemplate{footline}{sectioning}{%
  % default height is 0.4pt, which is ignored by adobe reader, so we increase it by 0.2pt
  {\usebeamercolor[fg]{separator line}\hrule height 0.6pt}%
  \hbox{%
  \begin{beamercolorbox}[wd=.8\paperwidth,ht=2.25ex,dp=1ex,left]{footline}%
    \usebeamerfont{footline}\kern\beamer@leftmargin\insertshorttitle
    \ifx\insertsection\@empty\else\qquad$\vartriangleright$\qquad\insertsection\fi
    \ifx\insertsubsection\@empty\else\qquad$\vartriangleright$\qquad\insertsubsection\fi
  \end{beamercolorbox}%
  \begin{beamercolorbox}[wd=.2\paperwidth,ht=2.25ex,dp=1ex,right]{footline}%
     \usebeamerfont{footline}\usebeamertemplate{footline right}%
     \kern\beamer@rightmargin
  \end{beamercolorbox}%
  }%
}

% customize mini frames template to get a section navigation bar

\defbeamertemplate{navigation box}{current}{%
  \colorbox{accent2}{%
    \rule[-1ex]{0pt}{3.25ex}\color{white}\kern1.4pt\my@navibox\kern1.4pt%
  }%
}

\defbeamertemplate{navigation box}{other}{%
  %\colorbox{white}{%
    \rule[-1ex]{0pt}{3.25ex}\color{black}\kern1.4pt\my@navibox\kern1.4pt%
  %}%
}

\newcommand{\my@navibox@subsection}{$\blacksquare$}
\newcommand{\my@navibox@frame}{$\square$}
\let\my@navibox=\my@navibox@frame

% optional navigation box for some special frame
\newcommand{\my@navibox@frame@opt}{$\boxplus$}
\newcommand{\my@change@navibox}{\let\my@navibox=\my@navibox@frame@opt}
\newcommand{\changenavibox}{%
  \addtocontents{nav}{\protect\headcommand{\protect\my@change@navibox}}%
}

\newcommand{\my@sectionentry@show}[5]{%
  \ifnum\c@section=#1%
    \setbeamertemplate{navigation box}[current]%
  \else
    \setbeamertemplate{navigation box}[other]%
  \fi
  \begingroup
    \def\my@navibox{#1}%
    \hyperlink{Navigation#3}{\usebeamertemplate{navigation box}}%
  \endgroup
}

\newif\ifmy@hidesection

\newcommand{\my@sectionentry@hide}[5]{\my@hidesectiontrue}

\pretocmd{\beamer@setuplinks}{\renewcommand{\beamer@subsectionentry}[5]{}}{}{}
\apptocmd{\beamer@setuplinks}{\global\let\beamer@subsectionentry\mybeamer@subsectionentry}{}{}

\newcommand{\mybeamer@subsectionentry}[5]{\global\let\my@navibox=\my@navibox@subsection}

\newcommand{\my@slideentry@empty}[6]{}

\newcommand{\my@slideentry@section}[6]{%
  \ifmy@hidesection
    \my@hidesectionfalse
  \else
    \ifnum\c@section=#1%
      \setbeamertemplate{navigation box}[other]%
      \ifnum\c@subsection=#2\ifnum\c@subsectionslide=#3%
         \setbeamertemplate{navigation box}[current]%
      \fi\fi
      \beamer@link(#4){\usebeamertemplate{navigation box}}%
    \fi
  \fi
  \global\let\my@navibox=\my@navibox@frame
}

\AtEndDocument{%
   \immediate\write\@auxout{%
     \noexpand\gdef\noexpand\my@totalsectionnumber{\the\c@section}%
   }%
}

\def\my@totalsectionnumber{0}

\defbeamertemplate{footline}{navigation}{%
  % default height is 0.4pt, which is ignored by adobe reader, so we increase it by 0.2pt
  {\usebeamercolor[fg]{separator line}\hrule height 0.6pt}%
  \begin{beamercolorbox}[wd=\paperwidth,ht=2.25ex,dp=1ex]{footline}%
    \usebeamerfont{footline}%
    \kern\beamer@leftmargin
    \setlength{\fboxsep}{0pt}%
    \ifnum\my@totalsectionnumber=0%
      \insertshorttitle
    \else
      \let\sectionentry=\my@sectionentry@show
      \let\slideentry=\my@slideentry@empty
      \dohead
    \fi
    \hfill
    \let\sectionentry=\my@sectionentry@hide
    \let\slideentry=\my@slideentry@section
    \dohead
    \kern\beamer@rightmargin
  \end{beamercolorbox}%
}

%% ------------------------- frame title -----------------------------

\defbeamertemplate{frametitle}{simple}[1][]
{%
  \nointerlineskip
  \begin{beamercolorbox}[wd=\paperwidth,sep=0pt,leftskip=\beamer@leftmargin,%
                         rightskip=\beamer@rightmargin,#1]{frametitle}
    \usebeamerfont{frametitle}%
    \rule[-3.6mm]{0pt}{12mm}\insertframetitle\rule[-3.6mm]{0pt}{12mm}\par
  \end{beamercolorbox}
}

%% ------------------- block and theorem -----------------------------

\defbeamertemplate{theorem begin}{simple}
{%
  \upshape%\bfseries\inserttheoremheadfont
  {\usebeamercolor[fg]{theoremname}%
  \inserttheoremname\inserttheoremnumber
  \ifx\inserttheoremaddition\@empty\else
    \ \usebeamercolor[fg]{local structure}(\inserttheoremaddition)%
  \fi%
  %\inserttheorempunctuation
  }%
  \quad\normalfont
}
\defbeamertemplate{theorem end}{simple}{\par}

\defbeamertemplate{proof begin}{simple}
{%
  %\bfseries
  \let\@addpunct=\@gobble
  {\usebeamercolor[fg]{proofname}\insertproofname}%
  \quad\normalfont
}
\defbeamertemplate{proof end}{simple}{\par}

%% ---------------------- enumerate and itemize ----------------------

\expandafter\patchcmd\csname beamer@@tmpop@enumerate item@square\endcsname
         {height1.85ex depth.4ex}{height1.85ex depth.3ex}{}{}
\expandafter\patchcmd\csname beamer@@tmpop@enumerate subitem@square\endcsname
         {height1.85ex depth.4ex}{height1.85ex depth.3ex}{}{}
\expandafter\patchcmd\csname beamer@@tmpop@enumerate subsubitem@square\endcsname
         {height1.85ex depth.4ex}{height1.85ex depth.3ex}{}{}

%% ------------------------ select templates -------------------------

\setbeamertemplate{background canvas}[default]
\setbeamertemplate{background}[line]
\setbeamertemplate{footline}[navigation]
\setbeamertemplate{footline left}[author]
\setbeamertemplate{footline center}[title]
\setbeamertemplate{footline right}[number]
\setbeamertemplate{title page}[banner]
\setbeamertemplate{title page content}[hexagon]
\setbeamertemplate{section page}[fill]
\setbeamertemplate{show sections in toc}[partial]
\setbeamertemplate{section name}[simple]
\setbeamertemplate{section name in toc}[simple]
\setbeamertemplate{section in toc}[fill]
\setbeamertemplate{section in toc shaded}[default][50]
\setbeamertemplate{subsection page}[fill]
\setbeamertemplate{subsection name}[simple]
\setbeamertemplate{subsection in toc}[fill]
\setbeamertemplate{subsection in toc shaded}[default][50]
\setbeamertemplate{theorem begin}[default]
\setbeamertemplate{theorem end}[default]
\setbeamertemplate{proof begin}[default]
\setbeamertemplate{proof end}[default]
\setbeamertemplate{frametitle}[simple]
\setbeamertemplate{navigation symbols}{}
\setbeamertemplate{itemize items}[square]
\setbeamertemplate{enumerate items}[square]

%% --------------------------- font theme ----------------------------

\setbeamerfont{title}{size=\LARGE}
\setbeamerfont{subtitle}{size=\large}
\setbeamerfont{author}{size=\normalsize}
\setbeamerfont{institute}{size=\normalsize}
\setbeamerfont{date}{size=\normalsize}
\setbeamerfont{occasion}{size=\normalsize}
\setbeamerfont{section in toc}{size=\large}
\setbeamerfont{subsection in toc}{size=\large}
\setbeamerfont{frametitle}{size=\large}
\setbeamerfont{block title}{size=\normalsize}
\setbeamerfont{item projected}{size=\footnotesize}
\setbeamerfont{subitem projected}{size=\scriptsize}
\setbeamerfont{subsubitem projected}{size=\tiny}

\usefonttheme{professionalfonts}
%\usepackage{arev}

%% ---------------------------- color theme --------------------------

% always use rgb colors in pdf files
\substitutecolormodel{hsb}{rgb}

\definecolor{red99}{Hsb}{0,0.9,0.9}
\definecolor{brown74}{Hsb}{30,0.7,0.4}
\definecolor{brown75}{Hsb}{30,0.7,0.5}
\definecolor{yellow86}{Hsb}{60,0.8,0.6}
\definecolor{yellow99}{Hsb}{60,0.9,0.9}
\definecolor{cyan95}{Hsb}{180,0.9,0.5}
\definecolor{blue67}{Hsb}{240,0.6,0.7}
\definecolor{blue74}{Hsb}{240,0.7,0.4}
\definecolor{blue77}{Hsb}{240,0.7,0.7}
\definecolor{blue99}{Hsb}{240,0.9,0.9}
\definecolor{magenta88}{Hsb}{300,0.8,0.8}

\colorlet{text1}{black}
\colorlet{back1}{white}
\colorlet{accent1}{blue99}
\colorlet{accent2}{cyan95}
\colorlet{accent3}{red99}
\colorlet{accent4}{yellow86}
\colorlet{accent5}{magenta88}
\colorlet{filler1}{accent1!40!back1}
\colorlet{filler2}{accent2!40!back1}
\colorlet{filler3}{accent3!40!back1}
\colorlet{filler4}{accent4!40!back1}
\colorlet{filler5}{accent5!40!back1}
\colorlet{gray1}{black!20}
\colorlet{gray2}{black!35}
\colorlet{gray3}{black!50}
\colorlet{gray4}{black!65}
\colorlet{gray5}{black!80}
\colorlet{tcolor}{text1!10!back1}
\colorlet{dcolor}{white}
\colorlet{fcolor}{blue77}
\colorlet{markcolor}{gray}
\colorlet{sectcolor}{brown74}

\setbeamercolor{normal text}{bg=white,fg=black}
\setbeamercolor{structure}{fg=blue99}
\setbeamercolor{local structure}{fg=cyan95}
\setbeamercolor{footline}{bg=,fg=black}
\setbeamercolor{title}{fg=yellow99}
\setbeamercolor{subtitle}{fg=white}
\setbeamercolor{author}{fg=black}
\setbeamercolor{institute}{fg=black}
\setbeamercolor{date}{fg=white}
\setbeamercolor{occasion}{fg=white}
\setbeamercolor{section name}{fg=brown75}
\setbeamercolor{section in toc}{fg=yellow99,bg=blue67}
\setbeamercolor{section in toc shaded}{fg=white,bg=blue74}
\setbeamercolor{subsection name}{parent=section name}
\setbeamercolor{subsection in toc}{use={structure,normal text},fg=structure.fg!90!normal text.bg}
\setbeamercolor{subsection in toc shaded}{parent=normal text}
\setbeamercolor{frametitle}{parent=structure}
\setbeamercolor{separator line}{fg=accent2}
\setbeamercolor{theoremname}{parent=subsection in toc}
\setbeamercolor{proofname}{parent=subsection in toc}
\setbeamercolor{block title}{fg=accent1,bg=gray}
\setbeamercolor{block body}{bg=lightgray}
\setbeamercolor{block title example}{fg=accent2,bg=gray}
\setbeamercolor{block body example}{bg=lightgray}
\setbeamercolor{block title alerted}{fg=accent3,bg=gray}
\setbeamercolor{block body alerted}{bg=lightgray}

%% ----------------------- handout mode ------------------------------

\mode<handout>{
  \setbeamertemplate{background canvas}{}
  \setbeamertemplate{background}[empty]
  \setbeamertemplate{footline}[sectioning]
  \setbeamertemplate{section page}[single]
  \setbeamertemplate{subsection page}[single]
  \setbeamerfont{subsection in toc}{size=\large}
  \colorlet{dcolor}{darkgray}
  \colorlet{fcolor}{white}
  \colorlet{sectcolor}{white}
  \setbeamercolor{normal text}{fg=black, bg=white}
  \setbeamercolor{title}{fg=blue}
  \setbeamercolor{subtitle}{fg=gray}
  \setbeamercolor{occasion}{fg=black}
  \setbeamercolor{date}{fg=black}
  \setbeamercolor{section in toc}{fg=blue!90!gray,bg=}
  \setbeamercolor{section in toc shaded}{fg=lightgray,bg=}
  \setbeamercolor{subsection in toc}{fg=blue!80!gray}
  \setbeamercolor{subsection in toc shaded}{fg=lightgray}
  \setbeamercolor{frametitle}{fg=blue!70!gray,bg=}
  \setbeamercolor{theoremname}{fg=blue!60!gray}
  \setbeamercolor{proofname}{fg=blue!60!gray}
  \setbeamercolor{footline}{bg=white,fg=black}
}

\mode
<all>

\makeatother

% -*- coding: utf-8 -*-

% ----------------------------------------------
% 中文显示相关代码
% ----------------------------------------------

% 以前要放在 usetheme 后面,否则报错;但是现在没问题了
\PassOptionsToPackage{CJKnumber}{xeCJK}
\usepackage[UTF8,noindent]{ctex}
%\usepackage[UTF8,indent]{ctexcap}

% 开明式标点:句末点号用全角,其他用半角。
%\punctstyle{kaiming}

% 在旧版本 xecjk 中用 CJKnumber 选项会自动载入 CJKnumb 包
% 但在新版本 xecjk 中 CJKnumber 选项已经被废弃,需要在后面自行载入它
\usepackage{CJKnumb}

%\CTEXoptions[today=big] % 数字年份前会有多余空白,中文年份前是正常的

\makeatletter
\ifxetex
  \setCJKsansfont{SimHei} % fix for ctex 2.0
  \renewcommand\CJKfamilydefault{\CJKsfdefault}%
\else
  \@ifpackagelater{ctex}{2014/03/01}{}{\AtBeginDocument{\heiti}} %无效?
\fi
\makeatother

%% 在旧版本 ctex 中,\today 命令生成的中文日期前面有多余空格
\makeatletter
\@ifpackagelater{ctex}{2014/03/01}{}{%
  \renewcommand{\today}{\number\year 年 \number\month 月 \number\day 日}
}
\makeatother

%% 在 xeCJK 中,默认将一些字符排除在 CJK 类别之外,需要时可以加入进来
%% 可以在 “附件->系统工具->字符映射表”中查看某字体包含哪些字符
% https://en.wikipedia.org/wiki/Number_Forms
% Ⅰ、Ⅱ、Ⅲ、Ⅳ、Ⅴ、Ⅵ、Ⅶ、Ⅷ、Ⅸ、Ⅹ、Ⅺ、Ⅻ
\xeCJKsetcharclass{"2150}{"218F}{1} % 斜线分数,全角罗马数字等
% https://en.wikipedia.org/wiki/Enclosed_Alphanumerics
\xeCJKsetcharclass{"2460}{"24FF}{1} % 带圈数字字母,括号数字字母,带点数字等

\ifxetex
% 在标点后,xeCJK 会自动添加空格,但不会去掉换行空格
%\catcode`,=\active  \def,{\textup{,} \ignorespaces}
%\catcode`;=\active  \def;{\textup{;} \ignorespaces}
%\catcode`:=\active  \def:{\textup{:} \ignorespaces}
%\catcode`。=\active  \def。{\textup{.} \ignorespaces}
%\catcode`.=\active  \def.{\textup{.} \ignorespaces}
\catcode`。=\active   \def。{.}
% 在公式中使用中文逗号和分号
%\let\douhao, \def\zhdouhao{\text{,\hskip-0.5em}}
%\let\fenhao; \def\zhfenhao{\text{;\hskip-0.5em}}
%\begingroup
%\catcode`\,=\active \protected\gdef,{\text{,\hskip-0.5em}}
%\catcode`\;=\active \protected\gdef;{\text{;\hskip-0.5em}}
% 似乎 beamer 的 \onslide<1,3> 不受影响
% 但是如果 tikz 图形中包含逗号,可能无法编译
%\catcode`\,=\active
%\protected\gdef,{\ifmmode\expandafter\zhdouhao\else\expandafter\douhao\fi}
%\catcode`\;=\active
%\protected\gdef;{\ifmmode\expandafter\zhfenhao\else\expandafter\fenhao\fi}
%\endgroup
%\AtBeginDocument{\catcode`\,=\active \catcode`\;=\active}
% 这样写反而无效
%\def\zhpunct{\catcode`\,=\active \catcode`\;=\active}
%\AtBeginDocument{%
%  \everymath\expandafter{\the\everymath\zhpunct}%
%  \everydisplay\expandafter{\the\everydisplay\zhpunct}%
%}
% 改为使用 kerkis 字体的逗号
\DeclareSymbolFont{myletters}{OML}{mak}{m}{it}
\SetSymbolFont{myletters}{bold}{OML}{mak}{b}{it}
\AtBeginDocument{%
  \DeclareMathSymbol{,}{\mathpunct}{myletters}{"3B}%
}
\fi

% 汉字下面加点表示强调
\usepackage{CJKfntef}

% ----------------------------------------------
% 字体选用相关代码
% ----------------------------------------------

% 虽然 arevtext 字体的宽度较大,但考虑到文档的美观还是同时使用 arevtext 和 arevmath
% 若在 ctex 包之前载入它,其设定的 arevtext 字体会在载入 ctex 时被重置为 lm 字体
% 因此我们在 ctex 宏包之后才载入它,但此时字体编码被改为 T1,需要修正 \nobreakspace
\usepackage{arev}
\DeclareTextCommandDefault{\nobreakspace}{\leavevmode\nobreak\ }

% 即使只需要 arevmath,也不能直接用 \usepackage{arevmath},
% 因为旧版本 fontspec 有问题,这样会导致它错误地修改数学字体

% 旧版本的 XeTeX 无法使用 arev sans 等 T1 编码字体的单独重音字符
% 因此我们恢复使用组合重音字符,见 t1enc.def, fntguide.pdf 和 encguide.pdf
\ifxetex\ifdim\the\XeTeXversion\XeTeXrevision pt<0.9999pt
  \DeclareTextAccent{\'}{T1}{1}
\fi\fi
% 在 T1enc.def 文件中定义了 T1 编码中的重音字符
% 先用 \DeclareTextAccent{\'}{T1}{1} 表示在 T1 编码中 \'e 等于 \accent"01 e
% 再用 \DeclareTextComposite{\'}{T1}{e}{233} 表示在 T1 编码中 \'e 等于 \char"E9

% ----------------------------------------------
% 版式定制相关代码
% ----------------------------------------------

\usepackage{hyperref}
\hypersetup{
  %pdfpagemode={FullScreen},
  bookmarksnumbered=true,
  unicode=true
}

%% 保证在新旧 ctex 宏包下编译得到相同的结果
\renewcommand{\baselinestretch}{1.3} % ctex 2.4.1 开始为 1,之前为 1.3

%% LaTeX 中 默认 \parskip=0pt plus 1pt,而 Beamer 中默认 \parskip=0pt

%% \parskip 用 plus 1fil 没有效果,用 plus 1fill 则节标题错位
\setlength{\parskip}{5pt plus 1pt minus 1pt}       % 段间距为 5pt + 1pt - 1pt
%\setlength{\baselineskip}{19pt plus 1pt minus 1pt} % 行间距为 5pt + 1pt - 1pt
\setlength{\lineskiplimit}{4pt}                    % 行间距小于 4pt 时重新设置
\setlength{\lineskip}{4pt}                         % 行间距太小时自动改为 4pt

\AtBeginDocument{
  \setlength{\baselineskip}{19pt plus 1pt minus 1pt} % 似乎不能放在导言区中
  \setlength{\abovedisplayskip}{4pt minus 2pt}
  \setlength{\belowdisplayskip}{4pt minus 2pt}
  \setlength{\abovedisplayshortskip}{2pt}
  \setlength{\belowdisplayshortskip}{2pt}
}

% 默认是 \raggedright,改为两边对齐
\usepackage{ragged2e}
\justifying
\let\oldraggedright\raggedright
\let\raggedright\justifying

% ----------------------------------------------
% 文本环境相关代码
% ----------------------------------------------

\setlength{\fboxsep}{0.02\textwidth}\setlength{\fboxrule}{0.002\textwidth}

\usepackage{adjustbox}
\newcommand{\mylinebox}[1]{\adjustbox{width=\linewidth}{#1}}

\usepackage{comment}
\usepackage{multicol}

% 带圈的数字
%\newcommand{\digitcircled}[1]{\textcircled{\raisebox{.8pt}{\small #1}}}
\newcommand{\digitcircled}[1]{%
  \tikz[baseline=(char.base)]{%
     \node[shape=circle,draw,inner sep=0.01em,line width=0.07em] (char) {\small #1};
  }%
}

\usepackage{pifont}
% 因为 xypic 将 \1 定义为 frm[o]{--},这里改为在文档内部定义
%\def\1{\ding{51}} % 勾
%\def\0{\ding{55}} % 叉

% 若在 enumerate 中使用自定义模板,则各项前的间距由第七项决定
% 对于我们使用的 arev 数学字体来说各个数字是等宽的,所以没问题
% 参考 https://tex.stackexchange.com/q/377959/8956
% 以及 https://chat.stackexchange.com/transcript/message/38541073#38541073
\newenvironment{enumskip}[1][]{%
  \setbeamertemplate{enumerate mini template}[default]%
  \if\relax\detokenize{#1}\relax % empty
    \begin{enumerate}[\quad(1)]
  \else
    \begin{enumerate}[#1][\quad(1)]
  \fi
}{\end{enumerate}}
\newenvironment{enumzero}[1][]{%
  \setbeamertemplate{enumerate mini template}[default]%
  \if\relax\detokenize{#1}\relax % empty
    \begin{enumerate}[(1)\,]
  \else
    \begin{enumerate}[#1][(1)\,]
  \fi
}{\end{enumerate}}
%
\newenvironment{enumlite}[1][]{%
  \setbeamertemplate{enumerate mini template}[default]%
  \setbeamercolor{enumerate item}{fg=,bg=}%
  \if\relax\detokenize{#1}\relax % empty
    \begin{enumerate}[(1)\,]%
  \else
    \begin{enumerate}[#1][(1)\,]%
  \fi
}{\end{enumerate}}
%
\newcounter{mylistcnt}
%
\newenvironment{enumhalf}{%
  \par\setcounter{mylistcnt}{0}%
  \def\item##1~{%
    \leavevmode\ifhmode\unskip\fi\linebreak[2]%
    \makebox[.5002\textwidth][l]{\stepcounter{mylistcnt}(\arabic{mylistcnt}) \,##1\ignorespaces}%
  }%
  \ignorespaces%
}{\par}
%
\newenvironment{choiceline}[1][]{%
  \par\vskip-0.5em\relax
  \setbeamertemplate{enumerate mini template}[default]%
  \setbeamercolor{enumerate item}{fg=,bg=}%
  \if\relax\detokenize{#1}\relax % empty
    \begin{enumerate}[(A)\,]
  \else
    \begin{enumerate}[#1][(A)\,]
  \fi
}{\end{enumerate}}
%
\newenvironment{choicehalf}{%
  \par\setcounter{mylistcnt}{0}%
  \def\item##1~{%
    \leavevmode\ifhmode\unskip\fi\linebreak[2]%
    \makebox[.5001\textwidth][l]{\stepcounter{mylistcnt}(\Alph{mylistcnt}) \,##1\ignorespaces}%
  }%
  \ignorespaces%
}{\par}
\newenvironment{choicequar}{%
  \par\setcounter{mylistcnt}{0}%
  \def\item##1~{%
    \leavevmode\ifhmode\unskip\fi\linebreak[0]%
    \makebox[.2501\textwidth][l]{\stepcounter{mylistcnt}(\Alph{mylistcnt}) \,##1\ignorespaces}%
  }%
  \ignorespaces%
}{\par}

% ----------------------------------------------
% 定理环境相关代码
% ----------------------------------------------

\makeatletter
\patchcmd{\@thm}{ \csname}{\kern0.18em\relax\csname}{}{}
\makeatother

\newcommand{\mynewtheorem}[2]{%
  \newtheorem{#1}{#2}[section]%
  \expandafter\renewcommand\csname the#1\endcsname{\arabic{#1}}%
}

\mynewtheorem{theorem}{定理}
\newtheorem*{theorem*}{定理}

\mynewtheorem{algorithm}{算法}
\newtheorem*{algorithm*}{算法}

\mynewtheorem{conjecture}{猜想}
\newtheorem*{conjecture*}{猜想}

\mynewtheorem{corollary}{推论}
\newtheorem*{corollary*}{推论}

\mynewtheorem{definition}{定义}
\newtheorem*{definition*}{定义}

\mynewtheorem{example}{例}
\newtheorem*{example*}{例子}

\mynewtheorem{exercise}{练习}
\newtheorem*{exercise*}{练习}

\mynewtheorem{fact}{事实}
\newtheorem*{fact*}{事实}

\mynewtheorem{guess}{猜测}
\newtheorem*{guess*}{猜测}

\mynewtheorem{lemma}{引理}
\newtheorem*{lemma*}{引理}

\mynewtheorem{method}{解法}
\newtheorem*{method*}{解法}

\mynewtheorem{origin}{引例}
\newtheorem*{origin*}{引例}

\mynewtheorem{problem}{问题}
\newtheorem*{problem*}{问题}

\mynewtheorem{property}{性质}
\newtheorem*{property*}{性质}

\mynewtheorem{proposition}{命题}
\newtheorem*{proposition*}{命题}

\mynewtheorem{puzzle}{题}
\newtheorem*{puzzle*}{题目}

\mynewtheorem{remark}{注记}
\newtheorem*{remark*}{注记}

\mynewtheorem{review}{复习}
\newtheorem*{review*}{复习}

\mynewtheorem{result}{结论}
\newtheorem*{result*}{结论}

\newtheorem*{analysis}{分析}
\newtheorem*{answer}{答案}
\newtheorem*{choice}{选择}
\newtheorem*{hint}{提示}
\newtheorem*{solution}{解答}
\newtheorem*{thinking}{思考}

\newcommand{\mynewtheoremx}[2]{%
  \newtheorem{#1}{#2}%
  \expandafter\renewcommand\csname the#1\endcsname{\arabic{#1}}%
}
\mynewtheoremx{bonus}{选做}
\newtheorem*{bonus*}{选做}

\renewcommand{\proofname}{证明}
\renewcommand{\qedsymbol}{}
\renewcommand{\tablename}{表格}

% ----------------------------------------------
% 数学环境相关代码
% ----------------------------------------------

% 选学内容的角标星号
\newcommand{\optstar}{\texorpdfstring{\kern0pt$^\ast$}{}}

\usepackage{mathtools} % \mathclap 命令
\usepackage{extarrows}

% 切换 amsmath 的公式编号位置
% 不使用 leqno 选项而在这里才修改,会导致编号与公式重叠
% 因此在 \documentclass 里都加上了 leqno 选项
\makeatletter
\newcommand{\leqnomath}{\tagsleft@true}
\newcommand{\reqnomath}{\tagsleft@false}
\makeatother
%\leqnomath

% 定义带圈数字的 tag 格式,需要 mathtools 包
\newtagform{circ}[\color{accent2}\digitcircled]{}{}
\newtagform{skip}[\quad\color{accent2}\digitcircled]{}{}

\newcounter{savedequation}

\newenvironment{aligncirc}{%
  \setcounter{savedequation}{\value{equation}}%
  \setcounter{equation}{0}%
  \usetagform{circ}%
  \align
}{
  \endalign
  \setcounter{equation}{\value{savedequation}}%
}
\newenvironment{alignskip}{%
  \setcounter{savedequation}{\value{equation}}%
  \setcounter{equation}{0}%
  \usetagform{skip}%
  \align
}{
  \endalign
  \setcounter{equation}{\value{savedequation}}%
}
\newenvironment{alignlite}{%
  \setcounter{savedequation}{\value{equation}}%
  \setcounter{equation}{0}%
  \align
}{
  \endalign
  \setcounter{equation}{\value{savedequation}}%
}

% cases 环境开始时 \def\arraystretch{1.2}
% 在中文文档中还有 \linespread{1.3},这样公式的左花括号就太大了
% 这里利用 etoolbox 包将 \linespread 临时改回 1
\AtBeginEnvironment{cases}{\linespread{1}\selectfont}
% 奇怪的是在最新的 miktex 中无此问题,
% 而且即使这样修改,在新旧 miktex 中用 arev 字体时花括号大小还是有差别
% 而用默认的 lm 字体时花括号却没有差别

% 用于给带括号的方程组编号
\usepackage{cases}

\newcommand{\R}{\mathbb{R}}
\newcommand{\Rn}{\mathbb{R}^n}

% 大型的积分号
\usepackage{relsize}
\newcommand{\Int}{\mathop{\mathlarger{\int}}}

% \oiint 命令
% \usepackage[integrals]{wasysym}

% http://tex.stackexchange.com/q/84302
\DeclareMathOperator{\arccot}{arccot}

% https://tex.stackexchange.com/a/178948/8956
% 保证 \d x 和 \d(2x) 和 \d^2 x 的间距都合适
\let\oldd=\d
\renewcommand{\d}{\mathop{}\!\mathrm{d}}
\newcommand{\dx}{\d x}
\newcommand{\dy}{\d y}
\def\dz{\d z} % 不确定命令是否已经定义
\newcommand{\du}{\d u}
\newcommand{\dv}{\d v}
\newcommand{\dr}{\d r}
\newcommand{\ds}{\d s}
\newcommand{\dt}{\d t}
\newcommand{\dS}{\d S}

\newcommand{\e}{\mathrm{e}}
\newcommand{\limit}{\lim\limits}

% 分数线长一点的分数,\wfrac[2pt]{x}{y} 表示左右加 2pt
% 参考 http://tex.stackexchange.com/a/21580/8956
\DeclareRobustCommand{\wfrac}[3][2pt]{%
  {\begingroup\hspace{#1}#2\hspace{#1}\endgroup\over\hspace{#1}#3\hspace{#1}}}

% 划去部分公式
% 横着划线,参考 http://tex.stackexchange.com/a/20613/8956
\newcommand{\hcancel}[2][accent3]{%
  \setbox0=\hbox{$#2$}%
  \rlap{\raisebox{.3\ht0}{\textcolor{#1}{\rule{\wd0}{1pt}}}}#2%
}
% 斜着划线,参考 https://tex.stackexchange.com/a/15958
\newcommand{\dcancel}[2][accent3]{%
    \tikz[baseline=(tocancel.base),ultra thick]{
        \node[inner sep=0pt,outer sep=0pt] (tocancel) {$#2$};
        \draw[#1] (tocancel.south west) -- (tocancel.north east);
    }%
}%

% 竖直居中的 \dotfill
\newcommand\cdotfill{\leavevmode\xleaders\hbox to 0.5em{\hss\footnotesize$\cdot$\hss}\hfill\kern0pt\relax}

% 保持居中的 \not 命令
\usepackage{centernot}

% 使用 stix font 中的 white arrows
\ifxetex
    \IfFileExists{STIX-Regular.otf}{%
        \newfontfamily{\mystix}{STIX} % stix v1.1
    }{%
        \newfontfamily{\mystix}{STIXGeneral} % stix v1.0
    }
    \DeclareRobustCommand\leftwhitearrow{%
      \mathrel{\text{\normalfont\mystix\symbol{"21E6}}}%
    }
    \DeclareRobustCommand\upwhitearrow{%
      \mathrel{\text{\normalfont\mystix\symbol{"21E7}}}%
    }
    \DeclareRobustCommand\rightwhitearrow{%
      \mathrel{\text{\normalfont\mystix\symbol{"21E8}}}%
    }
    \DeclareRobustCommand\downwhitearrow{%
      \mathrel{\text{\normalfont\mystix\symbol{"21E9}}}%
    }
\else
    \let \leftwhitearrow = \Leftarrow
    \let \rightwhitearrow = \Rightarrow
    \let \upwhitearrow = \Uparrow
    \let \downwhitearrow = \Downarrow
\fi

% ----------------------------------------------
% 绘图动画相关代码
% ----------------------------------------------

% pgf/tikz 的所有选项都称为 key,它们按照 unix 路径的方式组织,
% 例如:/tikz/external/force remake={boolean}
% 这些 key 可以用 \pgfkeys 定义,用 \tikzset 设置
% \tikzset 实际上等同于 \pgfkeys{/tikz/.cd,#1}.
% Using Graphic Options: P120 in manual 2.10 (\tikzset)
% Key Management: P481 in manual 2.10 (\pgfkeys)

\usepackage{tikz}
\usepackage{pgfplots}
%\usepackage{calc}

% 文档标注,通常需要编译两次就可以得到正确结果
% 但如果主题的 section page 用 tikz 绘图,将需要编译三次
% 这是因为 tikzmark 依赖 aux 文件的 pgfid 编号
% 第一次编译时缺少 toc 文件,将缺少若干个 tikz 图片
% 第二次编译时图片个数正确了,但是 aux 文件的 pgfid 仍然是错误的
% 这个问题在主题文件中已经修正了
\newcommand\tikzmark[1]{%
  \tikz[overlay,remember picture] \node[coordinate] (#1) {};%
}

% pgf 包含的  xxcolor 包存在问题,导致与 xeCJKfntef 包冲突
% 见 https://github.com/CTeX-org/ctex-kit/issues/323
% 注意此冲突在 ctex 2.9 中不存在,仅在最新的 miktex 2.9 中出现
\makeatletter
\g@addto@macro\XC@mcolor{\edef\current@color{\current@color}}
\makeatother

\usetikzlibrary{matrix} % 用于在 node 四周加括号
\usetikzlibrary{decorations}
\usetikzlibrary{decorations.markings} % 用于在箭头上作标记
\usetikzlibrary{intersections} % 用于计算路径的交点
\usetikzlibrary{positioning} % 可以更方便地定位
\usetikzlibrary{shapes.geometric} % 钻石形状节点

\usetikzlibrary{calc}
\usetikzlibrary{snakes}

% Externalizing Graphics: P343 and P651 in manual 2.10
\usetikzlibrary{external}
% 编译时需加上 --shell-escape(texlive)或 -enable-write18(miktex)选项
%\tikzexternalize[prefix=figure/] %\tikzexternalize[shell escape=-enable-write18]

% 默认 tikz 图片会用 pdflatex 编译,可以自己改为 xelatex
%\tikzset{external/system call={%
%  xelatex \tikzexternalcheckshellescape -halt-on-error -interaction=batchmode -jobname "\image" "\texsource"}}

% 强制重新生成图片, pgf 3.0 中会自动比较文件的 md5
%\tikzset{external/force remake} %\tikzset{external/remake next}

%\tikzset{draw=black,color=black}
%\mode<beamer>{\tikzset{every path/.style={color=white!90!black}}}

\usetikzlibrary{patterns}

% hack pgf prior to version 3.0 for pgf patterns in xetex
% code taken from pgfsys-dvipdfmx.def and pgfsys-xetex.def in pgf 3.0
\makeatletter
\def\myhackpgf{
  % fix typo in pgfsys-common-pdf-via-dvi.def in pgf 2.10
  \pgfutil@insertatbegineverypage{%
     \ifpgf@sys@pdf@any@resources%
        \special{pdf:put @resources
           << \ifpgf@sys@pdf@patterns@exists /Pattern @pgfpatterns \fi >>}%
     \fi%
  }
  % required to give colors on pattern objects.
  \pgfutil@addpdfresource@colorspaces{ /pgfprgb [/Pattern /DeviceRGB] }
  % hook for xdvipdfmx
  \def\pgfsys@dvipdfmx@patternobj##1{%
	 \pgfutil@insertatbegincurrentpagefrombox{##1}%
  }%
  % dvipdfmx provides a new special `pdf:stream' for a stream object
  \def\pgfsys@dvipdfmx@stream##1##2##3{%
     \special{pdf:stream ##1 (##2) << ##3 >>}%
  }%
  % declare patterns and set patterns
  \def\pgfsys@declarepattern##1##2##3##4##5##6##7##8##9{%
     \pgf@xa=##2\relax \pgf@ya=##3\relax%
     \pgf@xb=##4\relax \pgf@yb=##5\relax%
     \pgf@xc=##6\relax \pgf@yc=##7\relax%
     \pgf@sys@bp@correct\pgf@xa \pgf@sys@bp@correct\pgf@ya%
     \pgf@sys@bp@correct\pgf@xb \pgf@sys@bp@correct\pgf@yb%
     \pgf@sys@bp@correct\pgf@xc \pgf@sys@bp@correct\pgf@yc%
     \pgfsys@dvipdfmx@patternobj{%
        \pgfsys@dvipdfmx@stream{@pgfpatternobject##1}{##8}{%
           /Type /Pattern
           /PatternType 1
           /PaintType \ifnum##9=0 2 \else 1 \fi
           /TilingType 1
           /BBox [\pgf@sys@tonumber\pgf@xa\space\pgf@sys@tonumber\pgf@ya\space
                  \pgf@sys@tonumber\pgf@xb\space\pgf@sys@tonumber\pgf@yb]
           /XStep \pgf@sys@tonumber\pgf@xc\space
           /YStep \pgf@sys@tonumber\pgf@yc\space
           /Resources << >> %<<
        }%
     }%
     \pgfutil@addpdfresource@patterns{/pgfpat##1\space @pgfpatternobject##1}%
  }
  \def\pgfsys@setpatternuncolored##1##2##3##4{%
     \pgfsysprotocol@literal{/pgfprgb cs ##2 ##3 ##4 /pgfpat##1\space scn}%
  }
  \def\pgfsys@setpatterncolored##1{%
     \pgfsysprotocol@literal{/Pattern cs /pgfpat##1\space scn}%
  }
}
\@ifpackagelater{pgf}{2013/12/18}{}{\ifxetex\expandafter\myhackpgf\fi}%
\makeatother

% ----------------------------------------------
% 表格制作相关代码
% ----------------------------------------------

\newcommand{\narrowsep}[1][2pt]{\setlength{\arraycolsep}{#1}}
\newcommand{\narrowtab}[1][3pt]{\setlength{\tabcolsep}{#1}}

% diagbox 依赖 pict2e,但 miktex 中旧版本 pict2e 打包错误,使得引擎判别错误
% 从而导致在编译时出现大量警告,以及导致底栏右下角按钮链接错位
\ifxetex\PassOptionsToPackage{xetex}{pict2e}\fi
\usepackage{diagbox}

\usepackage{multirow} % 跨行表格

\usepackage{array} % 可以用 \extrarowheight
% 双倍宽度的横线和竖线,\arrayrulewidth 默认为 0.4pt
\setlength{\doublerulesep}{0pt}
\newcommand{\dhline}{\noalign{\global\arrayrulewidth0.8pt}\hline\noalign{\global\arrayrulewidth0.4pt}}
\newcolumntype{?}{!{\vrule width 0.8pt}} % 即使 \doublerulesep 为 0pt,|| 也不能得到双倍宽度
% 最好还是用 tabu,更简单

\usepackage{tabularx}

%\usepackage{arydshln} % 在分块矩阵中加虚线
%\setlength{\dashlinedash}{2pt} % 默认4pt
%\setlength{\dashlinegap}{2pt} % 默认4pt

% tabu 与 arydshln 会冲突,可以不使用 arydshln,
% 而用 tabu 定义虚线 \newcolumntype{:}{|[on 2pt off 2pt]}
% 参考 http://bbs.ctex.org/forum.php?mod=viewthread&tid=63944#pid405057
\usepackage{tabu}
\newcolumntype{:}{|[on 2pt off 2pt]}
\newcommand{\hdashline}{\tabucline[0.4pt on 2pt off 2pt]{-}} % 兼容 arydshln 的命令
\setlength{\tabulinesep}{4pt} % 拉开大型公式与表格横线的距离

%\usepackage{colortbl} % 否则 \taburowcolors 命令无效

% ----------------------------------------------
% 绝对定位相关代码
% ----------------------------------------------

\usepackage[absolute,overlay]{textpos}

% 将整个页面分为 32 乘 24 个边长为 4mm 的小正方形
\setlength{\TPHorizModule}{4mm}
\setlength{\TPVertModule}{4mm}

\setlength{\TPboxrulesize}{0.6pt}
\newlength{\tpmargin}
\setlength{\tpmargin}{2mm}

\newenvironment{bblock}[1][black]{%
  \begingroup
  \TPshowboxestrue\TPMargin{\tpmargin}%
  \textblockrulecolor{#1}\textblockcolour{}%
  \begin{textblock}%
}{%
  \end{textblock}%
  \endgroup
}
\newenvironment{cblock}[2][black]{%
  \begingroup
  \TPshowboxestrue\TPMargin{\tpmargin}%
  \textblockrulecolor{#1}\textblockcolour{#2}%
  \begin{textblock}%
}{%
  \end{textblock}%
  \endgroup
}
\newenvironment{cblocka}{\begin{cblock}{filler1}}{\end{cblock}}
\newenvironment{cblockb}{\begin{cblock}{filler2}}{\end{cblock}}
\newenvironment{cblockc}{\begin{cblock}{filler3}}{\end{cblock}}
\newenvironment{cblockd}{\begin{cblock}{filler4}}{\end{cblock}}
\newenvironment{cblocke}{\begin{cblock}{filler5}}{\end{cblock}}

% ----------------------------------------------
% 模版定制相关代码
% ----------------------------------------------

\usepackage{bookmark}

\newcommand{\mybookmark}[1]{%
  \bookmark[page=\thepage,level=3]{#1}%
  \changenavibox
}

%\setbeamercovered{transparent=5}

\setbeamersize{text margin left=4mm,text margin right=4mm}
\mode<beamer>{\setbeamertemplate{background}[linear]}
\setbeamertemplate{footline}[sectioning]
\setbeamertemplate{footline right}[normal]
\setbeamertemplate{theorem begin}[simple]
\setbeamertemplate{theorem end}[simple]
\setbeamertemplate{proof begin}[simple]
\setbeamertemplate{proof end}[simple]

% 段间距在 block 中也许无效 http://tex.stackexchange.com/q/6111/8956
%\addtobeamertemplate{block begin}{}{\setlength{\parskip}{6pt plus 2pt minus 2pt}}

%\mode<beamer>{\tikzset{every path/.style={color=black}}}

% 在 amsfonts.sty 中已经废弃 \bold 命令,改用 \mathbf 命令
\def\lead#1{\textcolor{accent1}{#1}}
\def\bold#1{\textcolor{accent2}{#1}}
\def\warn#1{\textcolor{accent3}{#1}}
\def\clead{\color{accent1}}
\def\cbold{\color{accent2}}
\def\cwarn{\color{accent3}}

\mode<handout>{
  \colorlet{filler1}{filler1!40!white}
  \colorlet{filler2}{filler2!40!white}
  \colorlet{filler3}{filler3!40!white}
  \colorlet{filler4}{filler4!40!white}
  \colorlet{filler5}{filler5!40!white}
  \colorlet{gray1}{gray1!60!white}
  \colorlet{gray2}{gray2!60!white}
  \colorlet{gray3}{gray3!60!white}
  \colorlet{gray4}{gray4!60!white}
  \colorlet{gray5}{gray5!60!white}
}

% 兼容性命令,在 beamer 中应该避免使用它们,而改用上面几个命令
\let\textbf=\bold \def\pmb{\usebeamercolor[fg]{local structure}}
\let\emph=\warn   \def\bm{\usebeamercolor[fg]{alerted text}}

\newcommand{\vpause}{\pause\vskip 0pt plus 0.5fill\relax}
\newcommand{\ppause}{\par\pause}

\newcommand{\mybackground}{\setbeamertemplate{background}[lattice][4mm]}
% 几个 \varxxx 命令是 arevmath 包提供的
% $\heartsuit\varheart\diamondsuit\vardiamond$
% $\varspade\spadesuit\varclub\clubsuit$
% rframe 为例题解答,sframe 为练习解答,可以选择不包含它们
\newenvironment{rframe}{\mybackground\begin{frame}}{\end{frame}}
\newenvironment{sframe}{%
  \mybackground
  \colorlet{markcolor}{accent4}%
  \backgroundmarklefttrue\backgroundmarkrighttrue
  \begin{frame}
}{\end{frame}}
\ifdefined\slide
  \setbeamertemplate{footline}[navigation]
  \renewenvironment{rframe}{\begin{frame}<beamer:0>}{\end{frame}}%
  \renewenvironment{sframe}{\begin{frame}<beamer:0>}{\end{frame}}%
\fi
\ifdefined\print
  \renewenvironment{sframe}{\begin{frame}<handout:0>}{\end{frame}}%
\fi
% 用于标示只针对内招或外招的内容:iframe 为内招,oframe 为外招
\newenvironment{iframe}{\backgroundmarklefttrue\begin{frame}}{\end{frame}}
\newenvironment{oframe}{\backgroundmarkrighttrue\begin{frame}}{\end{frame}}
\newenvironment{jframe}{\mybackground\backgroundmarklefttrue\begin{frame}}{\end{frame}}
\newenvironment{pframe}{\mybackground\backgroundmarkrighttrue\begin{frame}}{\end{frame}}
\def\myimode{i}
\def\myomode{o}
\ifx\slide\myimode
  \renewenvironment{oframe}{\begin{frame}<presentation:0>}{\end{frame}}%
  \renewenvironment{pframe}{\begin{frame}<presentation:0>}{\end{frame}}%
  \renewenvironment{jframe}{\begin{frame}<presentation:0>}{\end{frame}}%
\fi
\ifx\slide\myomode
  \renewenvironment{iframe}{\begin{frame}<presentation:0>}{\end{frame}}%
  \renewenvironment{jframe}{\begin{frame}<presentation:0>}{\end{frame}}%
  \renewenvironment{pframe}{\begin{frame}<presentation:0>}{\end{frame}}%
\fi
\ifx\print\myimode
  \renewenvironment{oframe}{\begin{frame}<presentation:0>}{\end{frame}}%
  \renewenvironment{pframe}{\begin{frame}<presentation:0>}{\end{frame}}%
\fi
\ifx\print\myomode
  \renewenvironment{iframe}{\begin{frame}<presentation:0>}{\end{frame}}%
  \renewenvironment{jframe}{\begin{frame}<presentation:0>}{\end{frame}}%
\fi

% 利用 tikzmark 作边注
\newcommand{\imark}[1][gray]{%
  \begin{tikzpicture}[overlay,remember picture]
    \node[coordinate] (A) {};
    \fill[color=#1] (current page.west |- A) rectangle +(1.2mm,0.6em);
  \end{tikzpicture}%
}
\newcommand{\omark}[1][gray]{%
  \begin{tikzpicture}[overlay,remember picture]
    \node[coordinate] (A) {};
    \fill[color=#1] (A -| current page.east) rectangle +(-1.2mm,0.6em);
  \end{tikzpicture}%
}
\newcommand{\smark}{%
  \imark[accent4]\omark[accent4]%
}
\newcommand{\itext}[1]{%
  \ifx\slide\myomode\else
    \ifx\print\myomode\else
      #1%
    \fi
  \fi
}
\newcommand{\otext}[1]{%
  \ifx\slide\myimode\else
    \ifx\print\myimode\else
      #1%
    \fi
  \fi
}
\newcommand{\stext}[1]{%
  \ifdefined\slide\else
    \ifdefined\print\else
      #1%
    \fi
  \fi
}

% 选择题的答案
\newcommand{\select}[1]{\qquad\stext{\llap{\makebox[2em]{\color{accent4}#1}}}}

%% 内外招同编号的定理,例子或练习等,需要将编号减一
\newcommand{\minusone}[1]{%
  \ifdefined\slide\else
    \ifdefined\print\else
      \addtocounter{#1}{-1}%
    \fi
  \fi
}

%\mode<beamer>{
%\def\mytoctemplate{
%  \setbeamerfont{section in toc}{size=\normalsize}
%  \setbeamerfont{subsection in toc}{size=\small}
%  \setbeamertemplate{section in toc shaded}[default][100]
%  \setbeamertemplate{subsection in toc}[subsections numbered]
%  \setbeamertemplate{subsection in toc shaded}[default][100]
%  \setbeamercolor{section in toc}{fg=structure.fg}
%  \setbeamercolor{section in toc shaded}{fg=structure.fg!50!black}
%  \setbeamercolor{subsection in toc}{fg=structure.fg}
%  \setbeamercolor{subsection in toc shaded}{fg=normal text.fg}
%  \begin{multicols}{2}
%  \tableofcontents[sectionstyle=show/shaded,subsectionstyle=show/shaded]
%  \end{multicols}
%}
%\AtBeginSection[]{\begin{frame}\frametitle{目录结构}\mytoctemplate\end{frame}}
%\AtBeginSubsection[]{\begin{frame}\frametitle{目录结构}\mytoctemplate\end{frame}}
%}

\mode<presentation>

\setbeamertemplate{section and subsection}[chinese]
\usebeamertemplate{section and subsection}

\mode
<all>

% -*- coding: utf-8 -*-

% ----------------------------------------------
% 高等数学中的定义和改动
% ----------------------------------------------

\newif\ifligong % 理工类或经济类
\ligongtrue

% Repeating Things: P504 in manual 2.10
\newcommand{\drawline}[4][]{%
  \foreach \v [remember=\v as \u,count=\i] in {#4} {
    \ifnum \i > 1
      \ifodd \i \draw[#1,#3] \u -- \v; \else \draw[#1,#2] \u -- \v; \fi
    \fi
  }
}
\newcommand{\drawplot}[5][]{%
  \foreach \v [remember=\v as \u,count=\i] in {#4} {
    \ifnum \i > 1
      \ifodd \i \draw[#1,#3] plot[domain=\u:\v] #5; \else \draw[#1,#2] plot[domain=\u:\v] #5; \fi
    \fi
  }
}

% http://tex.stackexchange.com/q/84302
\DeclareMathOperator{\Prj}{Prj}
\DeclareMathOperator{\grad}{grad}

\newcommand{\va}{\vec{a\vphantom{b}}}
\newcommand{\vb}{\vec{b}}
\newcommand{\vc}{\vec{c\vphantom{b}}}
\newcommand{\vd}{\vec{d}}
\newcommand{\ve}{\vec{e}}
\newcommand{\vi}{\vec{i}}
\newcommand{\vj}{\vec{j}}
\newcommand{\vk}{\vec{k}}
\newcommand{\vn}{\vec{n}}
\newcommand{\vs}{\vec{s}}
\newcommand{\vv}{\vec{v}}

\let\ov=\overrightarrow

% xcolor 支持 hsb 色彩模型,但 pgf 不支持,因此需要指定输出的色彩模型为 rgb
% 在 article 中可以用 \usepackage[rgb]{xcolor} \usepackage{tikz} 解决此问题
% 在 beamer 中可以用 \documentclass[xcolor={rgb}]{beamer} 解决此问题

%\definecolor{bcolor0}{Hsb}{0,0.6,0.9}   % red 红色
%\definecolor{bcolor1}{Hsb}{60,0.6,0.9}  % yellow 黄色
%\definecolor{bcolor2}{Hsb}{120,0.6,0.9} % green 绿色
%\definecolor{bcolor3}{Hsb}{180,0.6,0.9} % cyan 青色
%\definecolor{bcolor4}{Hsb}{240,0.6,0.9} % blue 蓝色
%\definecolor{bcolor5}{Hsb}{300,0.6,0.9} % magenta 洋红色

\colorlet{bcolor0}{accent3}
\colorlet{bcolor1}{accent1}
\colorlet{bcolor2}{accent2}
\colorlet{bcolor3}{accent4}
\colorlet{bcolor5}{accent5}


\begin{document}
%\ifxetex\else\heiti\fi

\occasion{高等数学课程}
\title{第十二章·无穷级数}
\author{\href{https://lvjr.bitbucket.io}{吕荐瑞}}
\institute{暨南大学数学系}

\begin{frame}[plain]
\titlepage
\end{frame}

\begin{frame}
\frametitle{无穷级数}
\begin{itemize}[<+->]
  \item $1+\dfrac12+\dfrac14+\dfrac18+\cdots+\dfrac1{2^{n-1}}+\cdots\onslide<+->{=2}$
  \item $1+\dfrac12+\dfrac13+\dfrac14+\cdots+\dfrac1n+\cdots\onslide<+->{=+\infty}$
  \item $1+\dfrac1{2^2}+\dfrac1{3^2}+\dfrac1{4^2}+\cdots+\dfrac1{n^2}+\cdots \onslide<+->{=\dfrac{\pi^2}6}$
  \item $1-\dfrac12+\dfrac13-\dfrac14+\cdots+\dfrac{(-1)^{n+1}}{n}+\cdots\onslide<+->{=\ln 2}$
  \item $1-\dfrac13+\dfrac15-\dfrac17+\cdots+\dfrac{(-1)^{n+1}}{2n-1}+\cdots\onslide<+->{=\dfrac{\pi}{4}}$
\end{itemize}
\end{frame}

\section{常数项级数的概念和性质}

\subsection{无穷级数的概念}

\begin{frame}
\frametitle{无穷级数}
\begin{definition}
给定数列:$u_1, u_2, u_3, \cdots, u_n, \cdots$,式子
\[u_1+u_2+u_3+\cdots+u_n+\cdots\]
称为\bold{无穷级数}(简称\bold{级数}),记为$\bold{\sum\limits_{n=1}^{\infty}u_n}$,\pause
其中第$n$项称为级数的\bold{通项}。
\end{definition}
\end{frame}

\begin{frame}
\frametitle{级数的敛散性}
级数$\sum\limits_{n=1}^{\infty}u_n$的前$n$项的和$S_n=u_1+u_2+\cdots+u_n$称为第$n$次\bold{部分和},\pause
各个部分和$S_1, S_2, \cdots, S_n, \cdots$ 构成一个数列。\pause
\begin{itemize}[<+->]
  \item 如果 $\limit_{n\to\infty}S_n=S$ 收敛,则称级数$\sum\limits_{n=1}^{\infty}u_n$ \bold{收敛};
  \begin{itemize}[<+->]
    \item 此时称$S$为级数的\bold{和},
    \item 称$R_n=S-S_n=u_{n+1}+u_{n+2}+\cdots$为级数的\bold{余项};
  \end{itemize}
  \item 如果 $\limit_{n\to\infty}S_n$ 发散,则称级数$\sum\limits_{n=1}^{\infty}u_n$ \bold{发散}。
\end{itemize}
\end{frame}

\begin{frame}
\begin{example}
讨论\bold{几何级数}(或称\bold{等比级数})
\[\sum\limits_{n=1}^{\infty}aq^{n-1}=a+aq+aq^2+\cdots+aq^{n-1}+\cdots\]
的敛散性,其中$a\neq0$,而$q$称为级数的\bold{公比}。
\end{example}
\end{frame}

\begin{frame}
\begin{example}
讨论级数$\sum\limits_{n=1}^{\infty}\frac1{n(n+1)}$的敛散性。
\end{example}
\pause
\begin{example}
讨论级数$\sum\limits_{n=1}^{\infty}\ln\frac{n+1}{n}$的敛散性。
\end{example}
\end{frame}

\subsection{收敛级数的性质}

\begin{frame}
\frametitle{无穷级数的运算:化正为负}
\noindent\begin{align*}
\cbold S&=1+\frac{1}2+\frac{1}4+\frac{1}8+\cdots
    = 1+\left(\frac{1}2+\frac{1}4+\frac{1}8+\cdots\right) \\
   &= 1 + \frac12\cdot(1+\frac{1}2+\frac{1}4+\cdots) = \bold{1+\frac12\,S}
\end{align*}
\onslide<2->{$\therefore\ \cbold S=2$}
\onslide<3->{\quad 即 \quad\lead{$\displaystyle 1+\frac12+\frac14+\frac18+\cdots=2$}}
\onslide<4->{\qquad\qquad\lead{\ding{51}}}
\onslide<5->{\par$\dotfill$}
\noindent\begin{align*}
%\sum_{n=1}^{\infty}2^{n-1}
\onslide<5->{\cbold T}
   &\onslide<5->{=1+2+4+8+\cdots = 1 + (2+4+8+\cdots)}\\
   &\onslide<5->{= 1 + 2\cdot(1+2+4+\cdots) = \bold{1+2T}}
\end{align*}
\noindent
\onslide<6->{$\therefore\ \cbold T=-1$}
\onslide<7->{\quad 即 \quad \cwarn{$1+2+4+8+\cdots=-1$}}
\onslide<8->{\qquad\quad{\cwarn\ding{55}}}
\end{frame}

\begin{frame}
\begin{property}
如果级数$\sum\limits_{n=1}^{\infty}u_n$和$\sum\limits_{n=1}^{\infty}v_n$都收敛,
则级数$\sum\limits_{n=1}^{\infty}(u_n\pm v_n)$也收敛,而且有
\[\bold{\sum\limits_{n=1}^{\infty}(u_n\pm v_n) = \sum\limits_{n=1}^{\infty}u_n \pm \sum\limits_{n=1}^{\infty}v_n}.\]
\end{property}
\end{frame}

\begin{frame}
\begin{property}
如果级数$\sum\limits_{n=1}^{\infty}u_n$收敛,则级数$\sum\limits_{n=1}^{\infty}au_n$也收敛,而且有
\[\bold{\sum\limits_{n=1}^{\infty}au_n = a\sum\limits_{n=1}^{\infty}u_n}.\]
\end{property}
\pause
\begin{corollary*}
级数的每一项同乘以不为$0$的常数后,其敛散性不变。
\end{corollary*}
\end{frame}

\begin{frame}
\begin{example}
求级数$\sum\limits_{n=1}^{\infty}\frac1{3^n}+\frac2{5^n}$的和。
\end{example}
\pause
\begin{example}
判定级数$\sum\limits_{n=1}^{\infty}3\ln\frac{n+1}n$的敛散性。
\end{example}
\end{frame}

\begin{frame}
\begin{property}
在一个级数前面加上(或者去掉)有限项,级数的敛散性不变。
\end{property}
\vpause
\begin{example}
设级数$\sum\limits_{n=1}^{\infty}u_n$的第$n$次部分和$S_n=\frac{n}{2n-1}$,
判断级数$\sum\limits_{n=1}^{\infty}u_{n+2}$的敛散性。若级数收敛,求出它的和。
\end{example}
\end{frame}

\begin{frame}
\frametitle{无穷级数的运算:无中生有}
\noindent\begin{align*}
  0 &= 0 + 0 + 0 + \cdots \\
    &= (1-1) + (1-1) + (1-1) + \cdots \\
    &= 1 - 1 + 1 - 1 + 1 - 1 + \cdots \text{\hspace{7em}\onslide<2->{\Large\cwarn\ding{55}}}\\
    &= 1 + (-1+1) + (-1+1) + (-1+1) + \cdots \\
    &= 1 + 0 + 0 + 0 + \cdots \\
    &= 1
\end{align*}
\end{frame}

\begin{frame}
\begin{property}[收敛级数的结合律]
如果一个级数收敛,加括号后所成的级数也收敛,且与原级数有相同的和。
\end{property}
\vpause
\begin{example}
已知几何级数
$$1+\dfrac12+\dfrac14+\dfrac18+\dfrac1{16}+\dfrac1{32}+\cdots=\dfrac1{1-1/2}=2.$$
\pause 加括号后得到的新级数
\begin{align*}
&\phantom{=}\left(1+\dfrac12\right)+\left(\dfrac14+\dfrac18\right)+\left(\dfrac1{16}+\dfrac1{32}\right)+\cdots\\
&\uncover<+->{=\frac32 + \frac38 + \frac3{32} + \cdots} \uncover<+->{= \frac{3/2}{1-1/4}=2.}
\end{align*}
\end{example}
\vfill\uncover<+->{
\begin{remark}
发散级数加括号后,可能发散也可能收敛。
\end{remark}}
\end{frame}

\begin{frame}
\frametitle{收敛的必要条件}
\begin{theorem}
如果级数$\sum\limits_{n=1}^{\infty}u_n$收敛,则有$\limit_{n\to\infty}u_n=0$。
\end{theorem}
\pause\dotfill
\begin{remark}
若通项不趋于零,则级数一定发散.
\end{remark}
\pause
\begin{example}
级数$\sum\limits_{n=1}^{\infty}\frac{n+1}{n}$的通项趋于$1$,因此它发散. 
\end{example}
\pause\dotfill
\begin{remark}
若通项趋于零,则级数\warn{未必}收敛.
\end{remark}
\pause
\begin{example}
级数$\sum\limits_{n=1}^{\infty}\ln\frac{n+1}{n}$的通项趋于$0$,但是它发散。
\end{example}
\end{frame}

\begin{frame}
\begin{exercise}
判断级数的敛散性.如果级数收敛,求出它的和。
\begin{enumlite}
  \item $\frac12+\frac34+\frac56+\frac78+\cdots+\frac{2n-1}{2n}+\cdots$
  \item $(\frac32+\frac13)+(\frac34+\frac19)+(\frac38+\frac1{27})+\cdots$
\end{enumlite}
\end{exercise}
\end{frame}

\mybookmark{复习与提高}

\begin{frame}
\frametitle{复习与提高}
\begin{puzzle}
判断级数$\sum\limits_{n=1}^{\infty}\dfrac{n}{2^n}$的敛散性,若收敛求其和.
\end{puzzle}
\stext{\pause\begin{solution}\smark
寻找$v_n$使得$\dfrac{n}{2^n}=v_n-v_{n+1}$.设$v_n=\dfrac{an+b}{2^n}$,
用待定系数法可以确定$a=b=2$.从而$S_n=v_1-v_{n+1}=2-\dfrac{n+2}{2^n}$,级数的和为$2$.
\end{solution}}
\end{frame}

\begin{frame}
\frametitle{复习与提高}
\begin{choice}
若级数$\sum\limits_{n=1}^{\infty}u_n$收敛,$\sum\limits_{n=1}^{\infty}v_n$发散,\vspace{-0.5em}%
则对于级数$\sum\limits_{n=1}^{\infty}(u_n\pm v_n)$来说有\dotfill(\select{B})
\begin{choicehalf}
  \item 级数收敛 ~
  \item 级数发散 ~
  \item 级数敛散性不定 ~
  \item 上述诸结论均不正确 ~
\end{choicehalf}
\end{choice}
\end{frame}

\section{常数项级数的审敛法}

\subsection{正项级数及其审敛法}

\begin{frame}
\begin{definition}
若级数$\sum\limits_{n=1}^{\infty}u_n$满足条件$u_n\ge0$(对所有$n$),则称它为\bold{正项级数}。
\end{definition}
\vpause
\begin{property*}
正项级数的部分和数列$\{S_n\}$是单调递增数列。
\end{property*}
\vpause
\begin{theorem}
正项级数收敛 $\Longleftrightarrow$ 它的部分和数列有界。
\end{theorem}
\vpause
\begin{remark*}
正项级数加括号后,其敛散性不变。
\end{remark*}
\end{frame}

%\subsubsection{比较判别法}

\begin{frame}
\begin{theorem}[比较判别法]
对于两个正项级数$\sum\limits_{n=1}^{\infty}u_n$和$\sum\limits_{n=1}^{\infty}v_n$,若有$c>0$使得$u_n\le cv_n$,
对所有$n$,则有
\begin{enumerate}
  \item 当$\sum\limits_{n=1}^{\infty}v_n$收敛时,$\sum\limits_{n=1}^{\infty}u_n$也收敛;
  \item 当$\sum\limits_{n=1}^{\infty}u_n$发散时,$\sum\limits_{n=1}^{\infty}v_n$也发散。
\end{enumerate}
\end{theorem}
\end{frame}

\begin{frame}
\frametitle{比较判别法}
\begin{example}
\bold{调和级数} $\sum\limits_{n=1}^{\infty}\dfrac1n$发散。
\end{example}
\vpause
\begin{example}
判断 \bold{$p$级数} $\sum\limits_{n=1}^{\infty}\dfrac1{n^p}$的敛散性。
\end{example}
\vpause
\begin{example}
判断级数$\sum\limits_{n=1}^{\infty}\dfrac1{n^n}$的敛散性。
\end{example}
\end{frame}

\begin{frame}
\begin{theorem}[比较判别法]
设$\sum\limits_{n=1}^{\infty}u_n$和$\sum\limits_{n=1}^{\infty}v_n$都为正项级数,
且有$\limit_{n\to\infty}\dfrac{u_n}{v_n}=l$。
\begin{enumerate}[<+->]
  \item 若$0<l<+\infty$,则$\sum\limits_{n=1}^{\infty}u_n$和$\sum\limits_{n=1}^{\infty}v_n$同敛散;
  \item 若$l=0$,则$\sum\limits_{n=1}^{\infty}v_n$收敛时,$\sum\limits_{n=1}^{\infty}u_n$也收敛;
  \item 若$l=+\infty$,则$\sum\limits_{n=1}^{\infty}v_n$发散时,$\sum\limits_{n=1}^{\infty}u_n$也发散。
\end{enumerate}
\end{theorem}
\end{frame}

\begin{frame}
\frametitle{比较判别法}
\begin{example}
判断级数$\sum\limits_{n=1}^{\infty}\dfrac1{2n-1}$的敛散性。
\end{example}
\vpause
\begin{example}
判断级数$\sum\limits_{n=1}^{\infty}\dfrac1{\sqrt{3n^2+n}}$的敛散性。
\end{example}
\vpause
\begin{example}
判断级数$\sum\limits_{n=1}^{\infty}\dfrac1{\sqrt{4n^3-3}}$的敛散性。
\end{example}
\vpause
\begin{example}
判断级数$\sum\limits_{n=1}^{\infty}\ln\left(1+\dfrac1{n^2}\right)$的敛散性。
\end{example}
\end{frame}

\begin{frame}
\frametitle{比较判别法}
\begin{exercise}
判断级数的敛散性。
\begin{enumlite}
  \item $\sum\limits_{n=1}^{\infty}\dfrac1{3n-2}$
  \item $\sum\limits_{n=1}^{\infty}\dfrac1{n\sqrt{n+1}}$
\end{enumlite}
\end{exercise}
\end{frame}

%\subsubsection{比值判别法}

\begin{frame}
\begin{theorem}[比值判别法]
如果正项级数$\sum\limits_{n=1}^{\infty}u_n$满足$\limit_{n\to\infty}\dfrac{u_{n+1}}{u_n}=l$,则有
\begin{enumerate}[<+->]
  \item 若$l<1$,则级数收敛;
  \item 若$l>1$,则级数发散;
  \item 若$l=1$,则级数可能收敛也可能发散。
\end{enumerate}
\end{theorem}
\end{frame}

\begin{frame}
\frametitle{比值判别法}
\begin{example}
设$x>0$,判定级数$\sum\limits_{n=1}^{\infty}\dfrac{x^n}{n}$的敛散性。
\end{example}
\vpause
\begin{example}
判定级数$\sum\limits_{n=1}^{\infty}\dfrac{n\cos^2\frac{n\pi}3}{2^n}$的敛散性。
\end{example}
\end{frame}

%\subsubsection{根值判别法}

\begin{frame}
\begin{theorem}[根值判别法]
如果正项级数$\sum\limits_{n=1}^{\infty}u_n$满足$\limit_{n\to\infty}\sqrt[n]{u_n}=\rho$,则有
\begin{enumerate}[<+->]
  \item 若$\rho<1$,则级数收敛;
  \item 若$\rho>1$,则级数发散;
  \item 若$\rho=1$,则级数可能收敛也可能发散。
\end{enumerate}
\end{theorem}
\end{frame}

\begin{frame}
\frametitle{根值判别法}
\begin{example}
设$a>0$,判定级数$\sum\limits_{n=1}^{\infty}\left(\dfrac{na}{n+1}\right)^n$的敛散性。
\end{example}
\end{frame}

\begin{frame}
\begin{exercise}
判定级数的敛散性:
\begin{enumlite}
  \item $\sum\limits_{n=1}^{\infty}\dfrac{2^n}{n(n+1)}$
  \item $\sum\limits_{n=1}^{\infty}\dfrac{3^n}{2^n(\arctan n)^n}$
\end{enumlite}
\end{exercise}
\end{frame}

\subsection{交错级数及其审敛法}

\begin{frame}
\frametitle{交错级数}
\begin{definition}
正负项相间的级数$\sum\limits_{n=1}^{\infty}(-1)^{n+1}u_n$,即
$$u_1-u_2+u_3-u_4+\cdots+u_{2k-1}-u_{2k}+\cdots,$$
其中每个$u_n>0$,称为\bold{交错级数}。
\end{definition}
\end{frame}

\begin{frame}
\frametitle{交错级数}
\begin{theorem}[莱布尼兹定理]
如果交错级数满足条件
\begin{enumerate}
  \item $u_{n+1}\le u_n$,$n=1,2,3,\cdots$;
  \item $\limit_{n\to\infty}u_n=0$;
\end{enumerate}
则级数收敛,且其和$S\le u_1$,余项满足$|R_n|\le u_{n+1}$。
\end{theorem}
\vpause
\begin{example}
交错级数$\sum\limits_{n=1}^{\infty}\dfrac{(-1)^{n+1}}n$收敛。
\end{example}
\end{frame}

\subsection{任意项级数的敛散性}

\begin{frame}
\frametitle{任意项级数}
\begin{theorem}
若$\sum\limits_{n=1}^{\infty}|u_n|$收敛,则$\sum\limits_{n=1}^{\infty}u_n$也收敛。
\end{theorem}
\pause
\begin{definition}
对任意项级数$\sum\limits_{n=1}^{\infty}u_n$,\pause
\begin{enumerate}[<+->]
  \item 称$\sum\limits_{n=1}^{\infty}u_n$ \bold{条件收敛},
        若$\sum\limits_{n=1}^{\infty}u_n$收敛,但$\sum\limits_{n=1}^{\infty}|u_n|$发散;
  \item 称$\sum\limits_{n=1}^{\infty}u_n$ \bold{绝对收敛},
        若$\sum\limits_{n=1}^{\infty}u_n$ 和 $\sum\limits_{n=1}^{\infty}|u_n|$ 都收敛。
\end{enumerate}
\end{definition}
\end{frame}

\begin{frame}
\frametitle{任意项级数}
\begin{theorem}
对于任意项级数$\sum\limits_{n=1}^{\infty}u_n$,若$\limit_{n\to\infty}\left|\dfrac{u_{n+1}}{u_n}\right|=l$,则有
\begin{enumerate}
  \item 当$l<1$时级数绝对收敛;
  \item 当$l>1$时级数发散。
\end{enumerate}
\end{theorem}
\end{frame}

\begin{frame}
\frametitle{任意项级数}
\begin{example}
判定级数是发散的,还是条件收敛的,还是绝对收敛的:
\begin{multicols}{2}
\begin{enumlite}[<+->]
  \item $\sum\limits_{n=1}^{\infty}(-1)^n\dfrac{n!}{n^n}$;
  \item $\sum\limits_{n=1}^{\infty}\dfrac{x^n}{n!}$;
  \item $\sum\limits_{n=1}^{\infty}\dfrac{x^n}{n}$;
  \item $\sum\limits_{n=1}^{\infty}nx^{n-1}$。
\end{enumlite}
\end{multicols}
\end{example}
\end{frame}

\begin{frame}
\frametitle{任意项级数}
\begin{exercise}
判定级数是发散的,还是条件收敛的,还是绝对收敛的:
\begin{enumlite}[<+->]
  \item $\sum\limits_{n=1}^{\infty}(-1)^{n+1}\dfrac{n}{2n-1}$;
  \item $\sum\limits_{n=1}^{\infty}(-1)^{n+1}\dfrac{1}{2n-1}$;
  \item $\sum\limits_{n=1}^{\infty}(-1)^{n+1}\dfrac{1}{n\cdot2^n}$。
\end{enumlite}
\end{exercise}
\end{frame}

\mybookmark{复习与提高}

\begin{frame}
\frametitle{复习与提高}
\begin{puzzle}
设正项级数$\sum\limits_{n=1}^{\infty}u_n$收敛,能否推出级数$\sum\limits_{n=1}^{\infty}u_n^2$也收敛?
\end{puzzle}
\stext{\pause\begin{solution}\smark
$\limit_{n\to\infty}\dfrac{u_n^2}{u_n}=\limit_{n\to\infty}u_n=0$,用比较判别法知级数收敛.
\end{solution}}
\end{frame}

\begin{frame}
\frametitle{复习与提高}
\begin{choice}
下列级数中收敛的是\dotfill(\select{C})
\begin{choicehalf}
  \item $\sum\limits_{n=1}^{\infty}\dfrac{1}{n^p}\ (0<p\le1)$  ~
  \item $\sum\limits_{n=1}^{\infty}\bigg(\dfrac{1}{n^2}-1\bigg)$ ~
  \item $\sum\limits_{n=1}^{\infty}\dfrac{1}{n(n+1)}$ ~
  \item $\sum\limits_{n=1}^{\infty}\bigg(\!-\dfrac{5}{2}\bigg)^n$ ~
\end{choicehalf}
\end{choice}
\end{frame}

\begin{frame}
\frametitle{复习与提高}
\begin{choice}
设$u_n\neq0$ $(n=1,2,3,\cdots)$,且$\limit_{n\to\infty}\dfrac{n}{u_n}=1$,则级数
$\sum\limits_{n=1}^{\infty}(-1)^{n+1}\left(\dfrac{1}{u_n}+\dfrac{1}{u_{n+1}}\right)$\dotfill(\select{C})
\begin{choicehalf}
  \item 发散 ~
  \item 绝对收敛 ~
  \item 条件收敛 ~
  \item 收敛性不能由条件确定 ~
\end{choicehalf}
\end{choice}
\end{frame}

\section{幂级数的概念}

\begin{frame}
\begin{definition}
形如$\sum\limits_{n=0}^{\infty}a_n(x-x_0)^n$的级数,即
\[ a_0 + a_1(x-x_0) + a_2(x-x_0)^2 + \cdots + a_n(x-x_0)^n + \cdots \]
称为 \bold{$x-x_0$的幂级数}。\ppause
特别地,当$x_0=0$时,级数$\sum\limits_{n=0}^{\infty}a_nx^n$,即
\[a_0+a_1x+a_2x^2+\cdots+a_nx^n+\cdots\]
称为 \bold{$x$的幂级数}。
\end{definition}
\end{frame}

\begin{frame}
\frametitle{幂级数的收敛域}
对于幂级数$\sum\limits_{n=0}^{\infty}a_nx^n$,
\begin{itemize}
  \item 若$x=x_0$时级数收敛,称$x_0$为幂级数的\bold{收敛点}\pause
  \item 若$x=x_0$时级数发散,称$x_0$为幂级数的\bold{发散点}
\end{itemize}
\pause
幂级数的全体收敛点构成的集合称为幂级数的\bold{收敛域}。
\end{frame}

\begin{frame}
%\frametitle{幂级数的收敛域}
\begin{theorem}
对幂级数$\sum\limits_{n=0}^{\infty}a_nx^n$,设$\limit_{n\to\infty}\left|\dfrac{a_{n+1}}{a_n}\right|=\rho$,则
\pause
\begin{enumerate}[<+->]
  \item 当$|x|<1/\rho$时,级数绝对收敛;
  \item 当$|x|>1/\rho$时,级数发散;
  \item 当$|x|=1/\rho$时,级数的敛散性未定。
\end{enumerate}
\end{theorem}
\vpause
\begin{definition*}
称$R=1/\rho$为幂级数的\bold{收敛半径},称$(-R,R)$为幂级数的\bold{收敛区间}。
\end{definition*}
\vpause
\begin{remark*}
当$\rho=0$时,规定$R=+\infty$;当$\rho=+\infty$时,规定$R=0$。
\end{remark*}
\end{frame}

\begin{frame}
\frametitle{幂级数的收敛域}
\begin{problem*}
给定幂级数$\sum\limits_{n=0}^{\infty}a_nx^n$,求出它的收敛域。
\end{problem*}
\begin{solution}
首先求出收敛半径$R$;
\begin{enumerate}[<+->]
  \item 若$0<R<+\infty$,则收敛域有四种可能
  \begin{multicols}{2}\begin{itemize}
    \item $(-R,R)$
    \item $[-R,R)$
    \item $(-R,R]$
    \item $[-R,R]$
  \end{itemize}\end{multicols}
  \item 若$R=0$,则收敛域为$\{0\}$;
  \item 若$R=+\infty$,则收敛域为$(-\infty,+\infty)$。
\end{enumerate}
\end{solution}
\end{frame}

\begin{frame}
\frametitle{幂级数的收敛域}
\begin{example}
求幂级数$\sum\limits_{n=0}^{\infty}\dfrac{(-1)^{n-1}x^n}n$的收敛域。
\end{example}
\vpause
\begin{example}
求幂级数$\sum\limits_{n=1}^{\infty}(-1)^{n-1}x^n$的收敛域。
\end{example}
\vpause
\begin{example}
求幂级数$\sum\limits_{n=0}^{\infty}\dfrac{x^n}{n!}$的收敛域。
\end{example}
\end{frame}

\begin{frame}
\frametitle{幂级数的收敛域}
\begin{example}
求幂级数$\sum\limits_{n=1}^{\infty}\dfrac{(2x+1)^n}n$的收敛域。
\end{example}
\pause
\begin{solution}
令$t=2x+1$。
\end{solution}
\vpause
\begin{example}
求幂级数$\sum\limits_{n=1}^{\infty}(-1)^{n-1}\dfrac{3^nx^{2n}}{n}$的收敛域。
\end{example}
\pause
\begin{solution}
令$t=x^2$或者令$t=3x^2$。
\end{solution}
\end{frame}

\begin{frame}
\frametitle{幂级数的收敛域}
\begin{exercise}
求幂级数$\sum\limits_{n=1}^{\infty}\dfrac{x^n}{(2n-1)(2n)}$的收敛域。
\end{exercise}
\end{frame}

\begin{frame}
\frametitle{幂级数的运算}
\begin{theorem*}
设$\sum\limits_{n=0}^{\infty}a_nx^n$和$\sum\limits_{n=0}^{\infty}b_nx^n$的收敛半径分别为$R_1$和$R_2$,
$R=\min\{R_1,R_2\}$,则有
\[ \bold{\big(\sum\limits_{n=0}^{\infty}a_nx^n\big)\pm
         \big(\sum\limits_{n=0}^{\infty}b_nx^n\big)
        =\sum\limits_{n=0}^{\infty}(a_n\pm b_n)x^n}, \]
其中等式在$(-R,R)$中成立。
\end{theorem*}
\end{frame}

\begin{frame}
\frametitle{幂级数的运算}
\begin{theorem*}
设$\sum\limits_{n=0}^{\infty}a_nx^n$和$\sum\limits_{n=0}^{\infty}b_nx^n$的收敛半径分别为$R_1$和$R_2$,
$R=\min\{R_1,R_2\}$,则有
\[ \bold{\big(\sum\limits_{n=0}^{\infty}a_nx^n\big)\cdot
         \big(\sum\limits_{n=0}^{\infty}b_nx^n\big)
        =\sum\limits_{n=0}^{\infty}c_nx^n}, \]
其中$c_n=\sum\limits_{k=0}^n a_kb_{n-k}$,等式在$(-R,R)$中成立。
\end{theorem*}
\end{frame}

\begin{frame}
\frametitle{幂级数的运算}
\begin{theorem*}
设$\sum\limits_{n=0}^{\infty}a_nx^n$和$\sum\limits_{n=0}^{\infty}b_nx^n$的收敛半径分别为$R_1$和$R_2$,
$R=\min\{R_1,R_2\}$,则有
\[ \bold{\big(\sum\limits_{n=0}^{\infty}a_nx^n\big)\Big/
         \big(\sum\limits_{n=0}^{\infty}b_nx^n\big)
        =\sum\limits_{n=0}^{\infty}d_nx^n}, \]
其中等式在$(-R,R)$的某个子区间内成立。
\end{theorem*}
\end{frame}

\begin{frame}
\begin{property}
幂级数$\sum\limits_{n=0}^{\infty}a_nx^n$的和函数$S(x)$在其收敛域上连续。
\end{property}
\vpause
\begin{property}
幂级数$\sum\limits_{n=0}^{\infty}a_nx^n$的和函数$S(x)$在其收敛域上可积,且有逐项积分公式
\begin{align*}
  \int_0^x S(x)\d x &=  \int_0^x \sum_{n=0}^{\infty}a_nx^n \d x \\
  &= \sum_{n=0}^{\infty}\int_0^xa_nx^n \d x = \sum\limits_{n=0}^{\infty}\frac{a_n}{n+1}x^{n+1}
\end{align*}
逐项积分后得到的幂级数和原级数有相同的收敛半径。
\end{property}
\end{frame}

\begin{frame}
\begin{property}
幂级数$\sum\limits_{n=0}^{\infty}a_nx^n$的和函数$S(x)$在其收敛区间$(-R,R)$上可导,且有逐项求导公式
\begin{align*}
  S'(x) &=  \big(\sum_{n=0}^{\infty}a_nx^n)' \\
  &= \sum_{n=0}^{\infty}\big(a_nx^n)' = \sum\limits_{n=0}^{\infty}na_nx^{n-1}
\end{align*}
逐项求导后得到的幂级数和原级数有相同的收敛半径。
\end{property}
\end{frame}

\begin{frame}
\begin{example}
对几何级数$\sum\limits_{n=0}^{\infty}x^n$逐项求导和逐项积分。
\end{example}
\itext{\vpause\begin{exercise}\imark
求无穷级数$\sum\limits_{n=1}^{\infty}\dfrac{n^2}{3^n}$的和.
\end{exercise}}
\end{frame}

\mybookmark{复习与提高}

\begin{frame}
\frametitle{复习与提高}
\begin{choice}%[2011年数学一]
设数列$\{a_n\}$单调减少,$\limit_{n\to\infty}a_n=0$,
部分和数列$S_n=\sum\limits_{k=1}^n a_n$($n=1,2,\cdots$)无界,
则幂级数$\sum\limits_{n=1}^{\infty}a_n(x-1)^n$的收敛域为\dotfill(\select{C})
\begin{choicehalf}
  \item $(-1,1]$  ~
  \item $[-1,1)$ ~
  \item $[0,2)$ ~
  \item $(0,2]$ ~
\end{choicehalf}
\end{choice}
\stext{\pause
\begin{solution}\smark
由莱布尼兹判别法$x=0$时级数收敛.另外$x=2$时级数发散.
\end{solution}}
\end{frame}

\section{函数展开成泰勒级数}

\subsection{泰勒公式和泰勒级数}

\begin{frame}
\begin{theorem*}[泰勒公式]
如果函数$f(x)$在包含$x_0$的区间$(a,b)$内有直到$n+1$阶的连续导数,
则当$x\in(a,b)$时,$f(x)$可按$x-x_0$的方幂展开为
\begin{align*}
  f(x)=f(x_0)&+f'(x_0)(x-x_0)+\frac{f''(x_0)}{2!}(x-x_0)^2\\
             &+\cdots+\frac{f^{(n)}(x_0)}{n!}(x-x_0)^n+R_n(x),
\end{align*}
\pause
其中余项$R_n(x)=\dfrac{f^{(n+1)}(\xi)}{(n+1)!}(x-x_0)^{n+1}$,%=o\big((x-x_0)^n)\big)
$\xi$介于$x_0$和$x$之间。
\end{theorem*}
\end{frame}

\begin{frame}
\frametitle{泰勒公式}
当$x_0=0$时,泰勒公式称为\bold{麦克劳林公式}
\begin{align*}
  f(x)=f(0)&+f'(0)x+\frac{f''(0)}{2!}x^2\\
             &+\cdots+\frac{f^{(n)}(0)}{n!}x^n+R_n(x),
\end{align*}
\pause
其中$R_n(x)=\dfrac{f^{(n+1)}(\xi)}{(n+1)!}x^{n+1}$,$\xi$介于$0$和$x$之间。
\ppause
令$\xi=\theta x$,则$R_n(x)=\dfrac{f^{(n+1)}(\theta x)}{(n+1)!}x^{n+1}$, $0<\theta<1$。
\end{frame}

\begin{frame}
%\frametitle{泰勒级数}
如果$f(x)$在区间$(a,b)$内各阶导数都存在,而且当$n\to\infty$时$R_n(x)\to0$,则得
\[f(x)=\sum\limits_{n=0}^{\infty}\frac{f^{(n)}(x_0)}{n!}(x-x_0)^n,\]
该级数称为函数$f(x)$的\bold{泰勒级数}。
\vpause
特别地,当$x_0=0$时,上式变成
\[f(x)=\sum\limits_{n=0}^{\infty}\frac{f^{(n)}(0)}{n!}x^n,\]
该级数称为函数$f(x)$的\bold{麦克劳林级数}。
\end{frame}

\subsection{初等函数的幂级数展开式}

\begin{frame}
\frametitle{直接展开法}
\begin{example}
求初等函数的幂级数展开式。
\begin{enumlite}[<+->]
  \item $f(x)=\e^x$
  \item $f(x)=\sin x$
  \item $f(x)=(1+x)^{\alpha}$
\end{enumlite}
\end{example}
\end{frame}

\begin{frame}
\frametitle{间接展开法}
\begin{example}
求初等函数的幂级数展开式。
\begin{enumlite}[<+->]
  \item $f(x)=\ln(1+x)$
  \item $f(x)=\arctan x$
  \item $f(x)=\cos x$
\end{enumlite}
\end{example}
\end{frame}

\begin{frame}[shrink=10]
\frametitle{初等函数的幂级数展开式}
\noindent\begin{align*}
e^x &= 1 + x + \frac{x^2}{2!} + \frac{x^3}{3!} + \cdots + \frac{x^n}{n!} + \cdots \\
\ln(1+x) &= x - \frac{x^2}2 + \frac{x^3}{3} - \frac{x^4}{4} + \cdots + (-1)^{n-1}\frac{x^n}{n} + \cdots \\
\sin x &= x-\frac{x^3}{3!}+\frac{x^5}{5!}-\frac{x^7}{7!}+\cdots + (-1)^n\frac{x^{2n+1}}{(2n+1)!} + \cdots \\
\cos x &= 1-\frac{x^2}{2!}+\frac{x^4}{4!}-\frac{x^6}{6!}+\cdots+(-1)^n\frac{x^{2n}}{(2n)!} + \cdots \\
\arctan x  &=x-\frac{x^3}{3}+\frac{x^5}{5}-\frac{x^7}{7}+\cdots + (-1)^n\frac{x^{2n+1}}{(2n+1)} + \cdots \\
(1+x)^{\alpha} &= 1 + C_\alpha^1x + C_\alpha^2x^2 + C_\alpha^3x^3 + \cdots + C_\alpha^n{x^n} + \cdots
\end{align*}
\end{frame}

\begin{frame}
\frametitle{初等函数的幂级数展开式}
\begin{example}
将函数$\e^{-x/3}$展成$x$的幂级数。
\end{example}
\vpause
\begin{example}
将函数$\sin^2x$展成$x$的幂级数。
\end{example}
\vpause
\begin{example}
将函数$\dfrac{x}{x+1}$展成$x$的幂级数。
\end{example}
\vpause
\begin{example}
将函数$\dfrac1{5-x}$展成$x-2$的幂级数。
\end{example}
\end{frame}

\begin{frame}
\frametitle{初等函数的幂级数展开式}
\begin{exercise}
将函数$\ln(1-x^2)$展成$x$的幂级数。
\end{exercise}
\pause
\begin{exercise}
将函数$\dfrac{x}{x+1}$展成$x-1$的幂级数。
\end{exercise}
\end{frame}

\begin{iframe}
\frametitle{初等函数的幂级数展开式}
\begin{example}
求$\arcsin x$的幂级数展开式。
\end{example}
\pause
\begin{solution}
由$(1+x)^\alpha$的幂级数展开式,可得
\[\frac1{\sqrt{1+x}}=1-\frac12x+\frac{1\cdot3}{2\cdot4}x^2-\frac{1\cdot3\cdot5}{2\cdot4\cdot6}x^3+\cdots\]
\pause
\[\frac1{\sqrt{1-x^2}}=1+\frac12x^2+\frac{1\cdot3}{2\cdot4}x^4+\frac{1\cdot3\cdot5}{2\cdot4\cdot6}x^6+\cdots\]
\pause
等式两边从$0$到$x$积分,即有
\[\arcsin x=x+\frac12\frac{x^3}3+\frac{1\cdot3}{2\cdot4}\frac{x^5}5
             +\frac{1\cdot3\cdot5}{2\cdot4\cdot6}\frac{x^7}7+\cdots\]
\end{solution}
\end{iframe}

\mybookmark{复习与提高}

\begin{frame}
\frametitle{复习与提高}
\begin{puzzle}
将函数$\ln(2+x-3x^2)$展开为$x$的幂级数.
\end{puzzle}
\vpause
\begin{puzzle}
将$f(x)=\arctan\dfrac{1+x}{1-x}$展开为$x$的幂级数.
\end{puzzle}
\stext{\pause\begin{solution}\smark
$f'(x)=\dfrac1{1+x^2}=\sum\limits_{n=0}^{\infty}(-1)^nx^{2n}$,$x\in(-1,1)$.
\[ f(x)-f(0) = \int_0^x \sum\limits_{n=0}^{\infty}(-1)^nx^{2n}\dx
   = \sum\limits_{n=0}^{\infty}\dfrac{(-1)^n}{2n+1}x^{2n+1} \]
另外$x=\pm1$时级数也收敛,故收敛区间为$[-1,1]$.
\end{solution}}
\end{frame}

\begin{frame}
\frametitle{复习与提高}
\begin{puzzle}
求幂级数$\sum\limits_{n=0}^{\infty}(-1)^n\dfrac{n+1}{(2n+1)!}x^{2n+1}$的和函数.
\end{puzzle}
\stext{\pause\begin{solution}\smark
$\dfrac12\sin x+\dfrac{x}2\cos x$,$x\in(-\infty,+\infty)$.
\end{solution}}
\end{frame}

\section{幂级数的应用}

\begin{frame}
\frametitle{幂级数的应用}
\begin{example}
计算$\e$的近似值。
\end{example}
\vpause
\begin{example}
计算$\pi$的近似值。
\end{example}
\vpause
\begin{example}
计算$\sqrt[5]{245}$的近似值。
\end{example}
\vpause
\begin{example}
计算$\int_0^{0.2}\e^{-x^2}\d x$的近似值。
\end{example}
\end{frame}

\begin{iframe}
\frametitle{幂级数的应用}
\begin{theorem*}%[欧拉]
$1+\dfrac{1}{3^2}+\dfrac{1}{5^2}+\cdots+\dfrac{1}{(2n-1)^2}+\cdots=\dfrac{\pi^2}8$。
\end{theorem*}
\pause
\vspace{0.5em}\hrule
%\begin{proof}
由$\arcsin x$的展开式及$\int_0^{\pi/2}\sin^{2n-1}t\d{t}$的公式有:\pause
\[\arcsin x=x+\frac12\frac{x^3}3+\frac{1\cdot3}{2\cdot4}\frac{x^5}5
             +\frac{1\cdot3\cdot5}{2\cdot4\cdot6}\frac{x^7}7+\cdots\] \pause
\[t=\sin t+\frac12\frac{\sin^3t}3+\frac{1\cdot3}{2\cdot4}\frac{\sin^5t}5
             +\frac{1\cdot3\cdot5}{2\cdot4\cdot6}\frac{\sin^7t}7+\cdots\] \pause
从$0$到$\dfrac{\pi}2$积分,得到
$\displaystyle\frac{\pi^2}8=1+\dfrac{1}{3^2}+\dfrac{1}{5^2}+\dfrac{1}{7^2}+\cdots$。
%\end{proof}
\end{iframe}

\begin{iframe}
%\frametitle{幂级数的应用}
\begin{theorem*}[欧拉]
$1+\dfrac{1}{2^2}+\dfrac{1}{3^2}+\cdots+\dfrac{1}{n^2}+\cdots=\dfrac{\pi^2}6$。
\end{theorem*}
\pause
\vspace{0.5em}\hrule
%\begin{proof}
\begin{align*}
\sum_{n=1}^{\infty}\frac1{n^2}&=1+\frac{1}{2^2}+\frac{1}{3^2}+\frac{1}{4^2}+\cdots+\dfrac{1}{n^2}+\cdots\\
&\uncover<3->{=\left(1+\frac{1}{3^2}+\frac{1}{5^2}+\cdots+\frac{1}{(2n-1)^2}+\cdots\right)}\\
&\uncover<3->{\qquad+\left(\frac{1}{2^2}+\frac{1}{4^2}+\frac{1}{6^2}+\cdots+\frac{1}{(2n)^2}+\cdots\right)}\\
&\uncover<4->{=\frac{\pi^2}8 + \frac14\sum_{n=1}^{\infty}\frac1{n^2}}
\uncover<5->{\qquad\qquad\Rightarrow\displaystyle\sum_{n=1}^{\infty}\frac1{n^2}=\frac{\pi^2}6}
\end{align*}
%\end{proof}
\end{iframe}

\ifligong % >>>>>>>>>>>>>>>>>>>>>>>>>>>>>>>>>>>>>>>>>>>>>>>>>>>>>>>>>>>>>>>>>>>>

\makeatletter
\beamer@tocsectionnumber=6\relax % 没有 \beamer@tocsubsectionnumber,修改 subsection 计数器时不用改
\setcounter{section}{6}
\makeatother

\section{傅里叶级数}

\begin{frame}
\frametitle{认识声波信号}
声音由物体的振动产生,如乐器演奏,声带振动.\pause
\begin{itemize}
  \item 基本的简谐振动产生正弦波$y=A\sin(\omega t+\varphi)$\pause
  \begin{itemize}
    \item 振幅\makebox[1.4em]{$A$}反映声音的\CJKunderdot{音量}%音强
    \item 频率\makebox[1.4em]{$\omega$}反映声音的\CJKunderdot{音调}%音高
  \end{itemize}\pause
  \item 复杂的物体振动的声波由不同频率的正弦波组成\pause
  \begin{itemize}
    \item 各频率正弦波的比例反映声音的\CJKunderdot{音色}%音质
  \end{itemize}
\end{itemize}\pause
人耳能感知的声音频率在20Hz至20000Hz之间,话音的频率在300Hz至3400Hz之间.\pause
\begin{multicols}{3}
\begin{itemize}
  \item 记录声音?\pause
  \item 压缩声音?\pause
  \item 去除噪音?
\end{itemize}
\end{multicols}
\end{frame}

\begin{rframe}
\frametitle{音频文件的波形}
许多音乐播放器都可以在播放时显示音频文件的波形.以 Windows Media Player 为例:
\begin{enumerate}
  \item 在音频文件的右键菜单选择\newline
        “打开方式 $\rightarrow$ Windows Media Player”
  \item 在“Windows Media Player”的右键菜单选择\newline
        “可视化效果 $\rightarrow$ 条形与波浪 $\rightarrow$ 波形”
\end{enumerate}
播放时可以随时暂停,并查看当前时刻的波形.
\end{rframe}

\subsection{三角级数与三角函数系}

\begin{frame}
%\frametitle{正弦函数}
正弦函数$y=A\sin(\omega t+\varphi)$具有周期$T=\frac{2\pi}{\omega}$,即
\[ A\sin(\omega t+\varphi)=A\sin(\omega(t+T)+\varphi) \]
\centering
\begin{tikzpicture}[scale=1.7,thick,font=\small]
%y=Asin(at+b) 
\def\A{0.8} 
\pgfmathsetmacro{\a}{3} 
\pgfmathsetmacro{\b}{pi/6}
%define domain 
\pgfmathsetmacro{\l}{2*pi/\a} 
\pgfmathsetmacro{\n}{1.65}
\pgfmathsetmacro{\d}{\l*\n}
%coordinates 
\draw[-stealth,thin,gray] (-\d,0)--(\d,0) node[above,color=text1] {$t$};
\draw[-stealth,thin,gray] (0,{-\A-0.3})--(0,{\A+0.6});
%help lines 
\draw[densely dashed,thin,gray] (-\d,\A)--(\d,\A);
\draw[fill] (0,\A) circle (0.02) node[below left,inner sep=1pt] {$A$}; 
\draw[densely dashed,thin,gray] (-\d,-\A)--(\d,-\A);
\draw[fill] (0,-\A) circle (0.02) node[below left,inner sep=1pt] {$-A$};
%y=Asin(at+b) 
\draw[thick] plot[variable=\t,domain=-\d:\d,samples=80,smooth] (\t,{\A*sin(\a*(\t+\b/\a) r)});
%period
\draw[blue,densely dashed] ({-\b/\a-3*pi/\a},-\A)--({-\b/\a-3*pi/\a},\A); 
\draw[blue,densely dashed] ({-\b/\a-pi/\a},-\A)--({-\b/\a-pi/\a},\A); 
\draw[blue,densely dashed] ({-\b/\a+pi/\a},-\A)--({-\b/\a+pi/\a},\A); 
\draw[blue,densely dashed] ({-\b/\a+3*pi/\a},-\A)--({-\b/\a+3*pi/\a},\A);
%
\draw[blue,densely dashed] ({-\b/\a-3*pi/\a},\A)--++(0,0.3); 
\draw[blue,densely dashed] ({-\b/\a-pi/\a},\A)--++(0,0.3); 
\draw[blue,densely dashed] ({-\b/\a+pi/\a},\A)--++(0,0.3); 
\draw[blue,densely dashed] ({-\b/\a+3*pi/\a},\A)--++(0,0.3);
%
\node[above,blue] at({-\b/\a+2*pi/\a},\A) (n1) {$T=\frac{2\pi}{\omega}$};
\node[above,blue] at({-\b/\a+0*pi/\a},\A) (n2) {$T=\frac{2\pi}{\omega}$};
\node[above,blue] at({-\b/\a-2*pi/\a},\A) (n3) {$T=\frac{2\pi}{\omega}$};
%
\node[above] at({-\b/\a+3*pi/\a},\A) (m1) {$\vphantom{T=\frac{2\pi}{\omega}}$};
\node[above] at({-\b/\a+1*pi/\a},\A) (m2) {$\vphantom{T=\frac{2\pi}{\omega}}$};
\node[above] at({-\b/\a-1*pi/\a},\A) (m3) {$\vphantom{T=\frac{2\pi}{\omega}}$};
\node[above] at({-\b/\a-3*pi/\a},\A) (m4) {$\vphantom{T=\frac{2\pi}{\omega}}$};
%
\draw[-stealth,blue] (n1)--(m1); 
\draw[-stealth,blue] (n1)--(m2); 
\draw[-stealth,blue] (n2)--(m2); 
\draw[-stealth,blue] (n2)--(m3);
\draw[-stealth,blue] (n3)--(m3); 
\draw[-stealth,blue] (n3)--(m4);
\end{tikzpicture}\ppause
设$n$为正整数,正弦函数$y=A_{n}\sin(n\omega t+\varphi_{n})$的最小正周期是$\frac{2\pi}{n\omega}$,
显然$T=\frac{2\pi}{\omega}$也是周期.
\end{frame}

\begin{frame}
\frametitle{周期函数}
设$f(t)$是定义在$\mathbb{R}$上的周期函数,周期也是$T=\frac{2\pi}{\omega}$.\par
{\centering
\begin{tikzpicture}[scale=1.7,thick,font=\small]
%y=Asin(at+b) 
\def\A{0.8} 
\pgfmathsetmacro{\a}{3} 
\pgfmathsetmacro{\b}{pi/6}
%define domain 
\pgfmathsetmacro{\l}{2*pi/\a} 
\pgfmathsetmacro{\n}{1.65}
\pgfmathsetmacro{\d}{\l*\n}
%coordinates 
\draw[-stealth,thin,gray] (-\d,0)--(\d,0) node[above,text1] {$t$};
\draw[-stealth,thin,gray] (0,{-\A})--(0,{\A+0.6});
%f(t)  
\draw[thick] plot[variable=\t,domain=-\d:\d,samples=80,smooth] (\t,{0.6*\A*sin(\a*(\t+\b/\a) r)+0.35*\A*sin(2*\a*\t r)});
%period
\draw[blue,densely dashed] ({-\b/\a-3*pi/\a},-\A)--({-\b/\a-3*pi/\a},\A); 
\draw[blue,densely dashed] ({-\b/\a-pi/\a},-\A)--({-\b/\a-pi/\a},\A); 
\draw[blue,densely dashed] ({-\b/\a+pi/\a},-\A)--({-\b/\a+pi/\a},\A); 
\draw[blue,densely dashed] ({-\b/\a+3*pi/\a},-\A)--({-\b/\a+3*pi/\a},\A);
%
\draw[blue,densely dashed] ({-\b/\a-3*pi/\a},\A)--++(0,0.3); 
\draw[blue,densely dashed] ({-\b/\a-pi/\a},\A)--++(0,0.3); 
\draw[blue,densely dashed] ({-\b/\a+pi/\a},\A)--++(0,0.3); 
\draw[blue,densely dashed] ({-\b/\a+3*pi/\a},\A)--++(0,0.3);
%
\node[above,blue] at({-\b/\a+2*pi/\a},\A) (n1) {$T=\frac{2\pi}{\omega}$};
\node[above,blue] at({-\b/\a+0*pi/\a},\A) (n2) {$T=\frac{2\pi}{\omega}$};
\node[above,blue] at({-\b/\a-2*pi/\a},\A) (n3) {$T=\frac{2\pi}{\omega}$};
%
\node[above] at({-\b/\a+3*pi/\a},\A) (m1) {$\vphantom{T=\frac{2\pi}{\omega}}$};
\node[above] at({-\b/\a+1*pi/\a},\A) (m2) {$\vphantom{T=\frac{2\pi}{\omega}}$};
\node[above] at({-\b/\a-1*pi/\a},\A) (m3) {$\vphantom{T=\frac{2\pi}{\omega}}$};
\node[above] at({-\b/\a-3*pi/\a},\A) (m4) {$\vphantom{T=\frac{2\pi}{\omega}}$};
%
\draw[-stealth,blue] (n1)--(m1); 
\draw[-stealth,blue] (n1)--(m2); 
\draw[-stealth,blue] (n2)--(m2); 
\draw[-stealth,blue] (n2)--(m3);
\draw[-stealth,blue] (n3)--(m3); 
\draw[-stealth,blue] (n3)--(m4);
\end{tikzpicture}}%
\ppause
\begin{problem*}
是否可将周期函数表示成正弦级数组成的级数
\[ f(t)=A_{0}+\sum_{n=1}^{\infty}A_{n}\sin(n\omega t+\varphi_{n})\ \warn{?} \]
\end{problem*}
\end{frame}

\begin{frame}
设$T=\frac{2\pi}{\omega}=2l$,则$f(t)$有周期区间为$[-l,l]$,\pause
%\begin{align*}
%    & A_{n}\sin(n\omega t+\varphi_{n})=A_{n}\sin\left(\frac{n\pi t}{l}+\varphi_{n}\right) \\
%={} & A_{n}\bigg[\sin\varphi_{n}\cos\frac{n\pi t}{l}+\cos\varphi_{n}\sin\frac{n\pi t}{l}\bigg] \\
%={} & a_{n}\cos\frac{n\pi t}{l}+b_{n}\sin\frac{n\pi t}{l}
%\end{align*}
%从而
\begin{align*}
f(t)&=A_{0}+\negthickspace\sum_{n=1}^{\infty}A_{n}\sin(n\omega t+\varphi_{n}) \\
&=\frac{a_{0}}{2}+\sum_{n=1}^{\infty}\bigg(\!a_{n}\cos\frac{n\pi t}{l}+b_{n}\sin\frac{n\pi t}{l}\bigg)
\end{align*}
\cdotfill\vpause
若周期$T=2\pi$($l=\pi$).$f(x)$有周期区间为$[-\pi,\pi]$,
\[
f(x)=\frac{a_{0}}{2}+\sum_{n=1}^{\infty}\bigg(a_{n}\cos nx+b_{n}\sin nx\bigg)
\]
\end{frame}

\begin{frame}
\begin{property*}
三角函数系 \bold{$1$, $\cos x$, $\sin x$, $\cos2x$, $\sin2x$, $\cdots$,
$\cos nx$, $\sin nx$, $\cdots$} 在区间$[-\pi,\pi]$上正交.\pause
即上述任何两个相异函数乘积,在$[-\pi,\pi]$上积分为零:
\[
\begin{array}{ll}
\int_{-\pi}^{\pi}\cos nx\dx=0,\quad\int_{-\pi}^{\pi}\sin nx\dx=0\quad(n\in\mathbb{N})_{\vphantom{\big|_{\big|}}}\\
\int_{-\pi}^{\pi}\sin kx\cdot\cos nx\dx=0\quad\hfill(k, n\in\mathbb{N},\ k\neq n)_{\vphantom{\big|_{\big|}}}\\
\int_{-\pi}^{\pi}\sin kx\cdot\sin nx\dx=0\quad\hfill(k, n\in\mathbb{N},\ k\neq n)_{\vphantom{\big|_{\big|}}}\\
\int_{-\pi}^{\pi}\cos kx\cdot\cos nx\dx=0\quad\hfill(k, n\in\mathbb{N},\ k\neq n)_{\vphantom{\big|_{\big|}}}
\end{array}
\]%
\noindent
\end{property*}
\vspace{-1em}\cdotfill\ppause
另外$\int_{-\pi}^{\pi}\sin^{2}nx\dx=\int_{-\pi}^{\pi}\cos^{2}nx\dx=\pi\quad(n\in\mathbb{N})$
\end{frame}

\subsection{函数展开为傅里叶级数}

\begin{frame}
\begin{theorem*}
若周期为$2\pi$的函数$f(x)$能展开为三角级数 
\[
f(x)=\frac{a_{0}}{2}+\sum_{n=1}^{\infty}\bigg(a_{n}\cos nx+b_{n}\sin nx\bigg)
\]
则有
\begin{align*}
a_{n}&=\frac{1}{\pi}\int_{-\pi}^{\pi}f(x)\cos nx\dx\quad(n=0,1,2,3,\cdots)\\
b_{n}&=\frac{1}{\pi}\int_{-\pi}^{\pi}f(x)\sin nx\dx\quad(n=1,2,3,\cdots)
\end{align*}
\end{theorem*}
\end{frame}

\begin{frame}
\begin{definition*}
$f(x)$的\bold{傅里叶级数}定义为
\[ \frac{a_{0}}{2}+\sum_{n=1}^{\infty}\bigg(a_{n}\cos nx+b_{n}\sin nx\bigg) \]
其中
\begin{align*}
a_{n}&=\frac{1}{\pi}\int_{-\pi}^{\pi}f(x)\cos nxdx\quad(n=0,1,2,3,\cdots)\\
b_{n}&=\frac{1}{\pi}\int_{-\pi}^{\pi}f(x)\sin nxdx\quad(n=0,1,2,3,\cdots)
\end{align*}
\end{definition*}
\begin{problem*}
何时有$\displaystyle f(x)\overset{\warn?}{=}\frac{a_{0}}{2}+\sum_{n=1}^{\infty}\big(a_{n}\cos nx+b_{n}\sin nx\big)$
\end{problem*}
\end{frame}

\begin{frame}
\begin{theorem*}[收敛定理]
设$f(x)$是周期为$2\pi$的周期函数,如果它满足:
\begin{enumerate}
  \item 在一个周期内连续或只有有限个第一类间断点
  \item 在一个周期内至多只有有限个极值点
\end{enumerate}
那么$f(x)$的傅里叶级数收敛,\pause 并且
\begin{enumerate}
  \item 当$x$是$f(x)$的连续点时,级数收敛于$f(x)$\pause
  \item 当$x$是$f(x)$的间断点时,级数收敛于\newline\hspace*{13em}
        $\frac{1}{2}\big[f(x^{-})+f(x^{+})\big]$
\end{enumerate}
\end{theorem*}
\end{frame}

\begin{frame}
\begin{example}
设$f(x)$是周期为$2\pi$的周期函数,在$[-\pi,\pi)$上的表达式为
\[
f(x)=\left\{\begin{array}{ll}
-1, & \quad-\pi\leq x<0,\\
1, & \quad0\leq x<\pi.
\end{array}\right.
\]
求出$f(x)$的傅里叶级数。
\end{example}
\pause\centering
\begin{tikzpicture}[scale=1.65,thick,font=\small]
%height 
\def\h{0.7}
%coordinates 
\draw[-stealth,thin,gray] (-3.1,0)--(3.1,0) node[right] {$x$}; 
\draw[-stealth,thin,gray] (0,-1)--(0,1) node[above] {$y$};
%f(t) 
\draw[fill] (0,\h) circle (0.5pt);
\draw (1,\h) circle (0.5pt);	
\draw[thick] (0,\h)--(1,\h);
\draw[fill] (-1,-\h) circle (0.5pt);
\draw (0,-\h) circle (0.5pt);
\draw[thick] (-1,-\h)--(0,-\h);
\node[below left] at(1,0) {$\pi$};
\node[below left] at(-1,0) {$-\pi$};
\node[below left] at(0,0) {$0$}; 
\node[left] at(0,\h) {$1$};
\node[below left] at(0,-\h) {$-1$};
\draw[densely dashed,thin,gray] (-1,-\h+0.05)--(-1,\h-0.05);
\draw[densely dashed,thin,gray] (1,-\h+0.05)--(1,\h-0.05);
\foreach \k in {-2,0,2}{
	\draw[fill] (\k,\h) circle (0.5pt);
	\draw (\k+1,\h) circle (0.5pt);
	\draw[thick] (\k,\h)--(\k+1,\h);
}
\foreach \k in {-3,-1,1}{ 
	\draw[fill] (\k,-\h) circle (0.5pt);
	\draw (\k+1,-\h) circle (0.5pt);
	\draw[thick] (\k,-\h)--(\k+1,-\h) ;
}
\foreach \k in {-3,-2,2,3} { 
	\node[below left] at(\k,0) {$\k\pi$};
}
%help lines 
\foreach \k in {-3,-2,...,3} { 
	\draw[densely dashed,thin,gray] (\k,-\h+0.05)--(\k,\h-0.05);
}
\end{tikzpicture}
\end{frame}

\begin{frame}
$f(x)$的傅里叶级数是$\displaystyle\frac{4}{\pi}\sum_{n=1}^{\infty}\frac{1}{2n-1}\sin\big[(2n-1)x\big]$\ppause
考虑级数的部分和,即前$N$项之和:\ppause
\centering 
\begin{tikzpicture}[scale=1.65,thick,font=\small,declare function={ 
	s1(\x)=4/pi*sin(\x r); 
	s2(\x)=s1(\x)+4/(3*pi)*sin(3*\x r);
	s3(\x)=s2(\x)+4/(5*pi)*sin(5*\x r);
	s4(\x)=s3(\x)+4/(5*pi)*sin(7*\x r);
	s5(\x)=s4(\x)+4/(9*pi)*sin(9*\x r);
	s6(\x)=s5(\x)+4/(11*pi)*sin(11*\x r);
	s7(\x)=s6(\x)+4/(13*pi)*sin(13*\x r);
	s8(\x)=s7(\x)+4/(15*pi)*sin(15*\x r);}]
%height 
\def\h{0.7}
%coordinates 
\draw[-stealth,black,thin,gray] (-3.1,0)--(3.1,0) node[right] {$x$}; 
\draw[-stealth,black,thin,gray] (0,-1)--(0,1) node[above] {$y$};
%f(t) 
\foreach \k in {-2,0,2}{
	\draw[fill] (\k,\h) circle (0.5pt);
	\draw (\k+1,\h) circle (0.5pt);
	\draw[thick] (\k,\h)--(\k+1,\h);
}
\foreach \k in {-3,-1,1}{ 
	\draw[fill] (\k,-\h) circle (0.5pt);
	\draw (\k+1,-\h) circle (0.5pt);
	\draw[thick] (\k,-\h)--(\k+1,-\h) ;
}
%help lines 
\foreach \k in {-3,-2,...,3} { 
	\draw[densely dashed,thin,gray] (\k,-\h+0.05)--(\k,\h-0.05);
}
\foreach \k in {-3,-2,2,3} { 
	\node[below left] at(\k,0) {$\k\pi$};
}
\node[below left] at(1,0) {$\pi$};
\node[below left] at(-1,0) {$-\pi$};
\node[below left] at(0,0) {$0$}; 
\node[left] at(0,\h) {$1$};
\node[below left] at(0,-\h) {$-1$};
%fourier series 
\only<4|handout:0>{
\node[blue] at (-1.5,\h+0.3) {$N=1$};
\draw[thick,blue] plot[variable=\t,domain=-3:3,samples=100,smooth] (\t,{\h*s1(\t*pi)});
}
\only<5|handout:0>{
\node[blue] at (-1.5,\h+0.3) {$N=2$};
\draw[thick,blue] plot[variable=\t,domain=-3:3,samples=300,smooth] (\t,{\h*s2(\t*pi)});
}
\only<6|handout:0>{
\node[blue] at (-1.5,\h+0.3) {$N=3$};
\draw[thick,blue] plot[variable=\t,domain=-3:3,samples=300,smooth] (\t,{\h*s3(\t*pi)});
}
\only<7|handout:0>{
\node[blue] at (-1.5,\h+0.3) {$N=4$};
\draw[thick,blue] plot[variable=\t,domain=-3:3,samples=300,smooth] (\t,{\h*s4(\t*pi)});
}
\only<8|handout:0>{
\node[blue] at (-1.5,\h+0.3) {$N=5$};
\draw[thick,blue] plot[variable=\t,domain=-3:3,samples=300,smooth] (\t,{\h*s5(\t*pi)});
}
\only<9|handout:0>{
\node[blue] at (-1.5,\h+0.3) {$N=6$};
\draw[thick,blue] plot[variable=\t,domain=-3:3,samples=300,smooth] (\t,{\h*s6(\t*pi)});
}
\only<10|handout:0>{
\node[blue] at (-1.5,\h+0.3) {$N=7$};
\draw[thick,blue] plot[variable=\t,domain=-3:3,samples=300,smooth] (\t,{\h*s7(\t*pi)});
}
\only<11->{
\node[blue] at (-1.5,\h+0.3) {$N=8$};
\draw[thick,blue] plot[variable=\t,domain=-3:3,samples=300,smooth] (\t,{\h*s8(\t*pi)});
}
\end{tikzpicture}
\end{frame}

\begin{frame}
\begin{example}
设$f(x)$是周期为$2\pi$的周期函数,在$[-\pi,\pi)$上的表达式为
\[ f(x)=|x| \]
求出$f(x)$的傅里叶级数。
\end{example}
\pause\centering
\begin{tikzpicture}[scale=1.65,thick,font=\small]
%height
\def\h{1}
%coordinates 
\draw[-stealth,black,thin,gray] (-3.1,0)--(3.1,0) node[right] {$x$};
\draw[-stealth,black,thin,gray] (0,-0.3)--(0,1.3) node[above] {$y$};
%f(t) 
\draw[thick] (-1,\h)--(0,0)--(1,\h);
\node[below] at(1,0) {$\pi$};
\node[below] at(-1,0) {$-\pi$};
\node[below left] at(0,0) {$0$}; 
\node[above left] at(0,\h) {$\pi$};
\draw[densely dashed,thin,gray] (-3,\h)--(3,\h);
\draw[densely dashed,thin,gray] (-1,0)--(-1,\h);
\draw[densely dashed,thin,gray] (1,0)--(1,\h);
\visible<3->{
\foreach \k in {-3,-1,1}{  
	\draw[thick] (\k,\h)--(\k+1,0)--(\k+2,\h);
}
\foreach \k in {-3,-2,2,3} { 
	\node[below] at(\k,0) {$\k\pi$};
}
}
\end{tikzpicture}
\end{frame}

\begin{frame}
$f(x)$的傅里叶级数是
$\displaystyle\frac{\pi}{2}-\frac{4}{\pi}\sum_{n=1}^{\infty}\frac{\cos[(2n-1)x]}{(2n-1)^{2}}$\ppause
考虑级数的部分和,即前$N$项之和:\ppause
\centering 
\begin{tikzpicture}[scale=1.65,thick,font=\small,declare function={
 s1(\x)=pi/2-4/pi*cos(\x r);
 s2(\x)=s1(\x)-4/(3*3*pi)*cos(3*\x r);
 s3(\x)=s2(\x)-4/(5*5*pi)*cos(5*\x r); 	
 s4(\x)=s3(\x)-4/(7*7*pi)*cos(7*\x r);
 s5(\x)=s4(\x)-4/(9*9*pi)*cos(9*\x r);
 s6(\x)=s5(\x)-4/(11*11*pi)*cos(11*\x r);
 s7(\x)=s6(\x)-4/(13*13*pi)*cos(13*\x r);
 s8(\x)=s7(\x)-4/(15*15*pi)*cos(15*\x r);}]
%height
\def\h{1}
%coordinates 
\draw[-stealth,black,thin,gray] (-3.1,0)--(3.1,0) node[right] {$x$};
\draw[-stealth,black,thin,gray] (0,-0.3)--(0,1.3) node[above] {$y$};
%f(t) 
\foreach \k in {-3,-1,1}{  
	\draw[thick] (\k,\h)--(\k+1,0)--(\k+2,\h);
}
\foreach \k in {-3,-2,2,3} { 
	\node[below] at(\k,0) {$\k\pi$};
}
\node[below] at(1,0) {$\pi$};
\node[below] at(-1,0) {$-\pi$}; 
\node[below left] at(0,0) {$0$}; 
%fourier series 
\only<4|handout:0>{
\node[blue] at (-1.5,\h+0.3) {$N=1$};
\draw[thick,blue] plot[variable=\t,domain=-3:3,samples=100,smooth] (\t,{\h/pi*s1(\t*pi)});
}
\only<5>{
\node[blue] at (-1.5,\h+0.3) {$N=2$};
\draw[thick,blue] plot[variable=\t,domain=-3:3,samples=200,smooth] (\t,{\h/pi*s2(\t*pi)});
}
\only<6|handout:0>{
\node[blue] at (-1.5,\h+0.3) {$N=3$};
\draw[thick,blue] plot[variable=\t,domain=-3:3,samples=200,smooth] (\t,{\h/pi*s3(\t*pi)});
}
\only<7|handout:0>{
\node[blue] at (-1.5,\h+0.3) {$N=4$};
\draw[thick,blue] plot[variable=\t,domain=-3:3,samples=200,smooth] (\t,{\h/pi*s4(\t*pi)});
}
\only<8|handout:0>{
\node[blue] at (-1.5,\h+0.3) {$N=5$};
\draw[thick,blue] plot[variable=\t,domain=-3:3,samples=200,smooth] (\t,{\h/pi*s5(\t*pi)});
}
\end{tikzpicture}
\end{frame}

\subsection{正弦级数和余弦级数}

\begin{frame}
\frametitle{正弦级数和余弦级数}
\begin{property*}
设$f(x)$是周期为$2\pi$的周期函数,
\begin{itemize}
  \item 若$f(x)$是奇函数,则傅里叶级数为\bold{正弦级数}
    \[ \sum_{n=1}^{\infty}b_{n}\sin nx,\quad b_{n}=\frac{2}{\pi}\int_{0}^{\pi}f(x)\sin nx\dx \]\pause
  \item 若$f(x)$是偶函数,则傅里叶级数为\bold{余弦级数}
    \[ \frac{a_{0}}{2}+\sum_{n=1}^{\infty}a_{n}\cos nx,\quad a_{n}=\frac{2}{\pi}\int_{0}^{\pi}f(x)\cos nx\dx \]
\end{itemize}
\end{property*}
\end{frame}

\begin{frame}
\frametitle{周期延拓}
设$f(x)$是定义在区间$[-\pi,\pi)$(或$(-\pi,\pi]$)上的函数,可以对其进行\bold{周期延拓},
从而得到定义在$\mathbb{R}$上的周期函数:\ppause{\centering
\begin{tikzpicture}[scale=1.65,thick,font=\small,declare function={ 
	f(\x)=0.8*(cos(pi/2*\x r)+1/4*cos(4*pi/2*\x r));}]
%coordinates 
\draw[-stealth,black,thin,gray] (-3.1,0)--(3.1,0) node[right] {$x$}; 
\draw[-stealth,black,thin,gray] (0,-0.9)--(0,1.1) node[above] {$y$};
%f(t) 
\draw[thick] plot[variable=\x,domain=-1:1,samples=50,smooth](\x,{f(\x+1)});
\draw[thin,gray,densely dashed] (-1,{f(0)})--(-1,{f(2)});
\draw[thin,gray,densely dashed] (1,{f(0)})--(1,{f(2)});
\draw[fill] (-1,{f(0)}) circle (0.5pt); 
\draw (1,{f(2)}) circle (0.6pt);
\node[below left] at(1,0) {$\pi$};
\node[below left] at(-1,0) {$-\pi$}; 
\node[below left] at(0,0) {$0$};
\foreach \k in {-3,-1,1}{  
	\draw[thick] plot[variable=\x,domain={\k}:{\k+2},samples=50,smooth](\x,{f(\x-\k)});
}
\foreach \k in {-3,3} { 
	\node[below left] at(\k,0) {$\k\pi$}; 
}
\foreach \k in {-3,-1,1,3} { 
	\draw[thin,gray,densely dashed] (\k,{f(0)})--(\k,{f(2)});
}
\foreach \k in {-3,-1,1} { 
	\draw[fill] (\k,{f(0)}) circle (0.5pt); 
}
\foreach \k in {-1,1,3} { 
	\draw (\k,{f(2)}) circle (0.6pt);
}
\end{tikzpicture}}
\ppause 延拓后的周期函数仍记为$f(x)$,此时可作傅里叶展开。
\end{frame}


\begin{frame}
\frametitle{奇延拓}
设$f(x)$是定义在区间$(0,\pi]$上的函数,可以对其进行\bold{奇延拓},
从而得到定义在$\mathbb{R}$上的周期奇函数。\ppause
\centering 
\begin{tikzpicture}[scale=1.65,thick,font=\small,declare function={ 
	f(\x)=0.6*0.8*sin(3*(\x+pi/18) r)+0.35*0.8*sin(2*3*\x r);}]
%coordinates 
\draw[-stealth,black,thin,gray] (-3.1,0)--(3.1,0) node[right] {$x$}; 
\draw[-stealth,black,thin,gray] (0,-0.9)--(0,0.9) node[left] {$y$};
%f(t) 
\draw[thick] plot[variable=\x,domain=0:1,samples=50,smooth](\x,{f(\x)});
\draw[thin,gray,densely dashed] (1,0)--(1,{f(1)});
\draw[fill] (1,{f(1)}) circle (0.7pt); 
\draw (0,{f(0)}) circle (0.7pt);
\node[below left,inner sep=1pt] at(1,0) {$\pi$}; 
\node[below left,inner sep=1pt] at(0,0) {$0$};
\visible<3->{
\draw[fill] (0,0) circle (0.7pt);
}
\visible<4->{
\draw[thick] plot[variable=\x,domain=-1:0,samples=50,smooth](\x,{-f(-\x)});
\draw[thin,gray,densely dashed] (-1,0)--(-1,{-f(1)});
\node[below left,inner sep=1pt] at(-1,0) {$-\pi$};
\draw (-1,{-f(1)}) circle (0.7pt); 
\draw (0,{-f(0)}) circle (0.7pt);
}
\visible<5->{
\foreach \k in {-3,-1,1}{  
	\draw[thick] plot[variable=\x,domain=-1:0,samples=50,smooth](\x+\k+1,{-f(-\x)});
	\draw (\k,{-f(1)}) circle (0.7pt); 
	\draw (\k+1,{-f(0)}) circle (0.7pt);
	\draw[thick] plot[variable=\x,domain=0:1,samples=50,smooth](\x+\k+1,{f(\x)});
	\draw[fill] (\k+2,{f(1)}) circle (0.7pt); 
	\draw (\k+1,{f(0)}) circle (0.7pt);
	\draw[fill] (\k+1,0) circle (0.7pt);
}
\draw[thin,gray,densely dashed] (-3,{f(1)})--(-3,{-f(1)});
\draw[thin,gray,densely dashed] (3,{f(1)})--(3,{-f(1)});
\draw[thin,gray,densely dashed] (-1,0)--(-1,{f(1)});
\draw[thin,gray,densely dashed] (1,0)--(1,{-f(1)});
\foreach \k in {-3,3} { 
	\node[below left,inner sep=1pt] at(\k,0) {$\k\pi$}; 
}
}
\end{tikzpicture} 
\end{frame}

\begin{frame}
\frametitle{偶延拓}
设$f(x)$是定义在区间$[0,\pi]$上的函数,可以对其进行\bold{偶延拓},
从而得到定义在$\mathbb{R}$上的周期偶函数。\ppause
\centering
\begin{tikzpicture}[scale=1.65,thick,font=\small,declare function={ 
	f(\x)=0.6*0.8*sin(3*(\x+pi/18) r)+0.35*0.8*sin(2*3*\x r);}]
%coordinates 
\draw[-stealth,black,thin,gray] (-3.1,0)--(3.1,0) node[right] {$x$}; 
\draw[-stealth,black,thin,gray] (0,-0.9)--(0,0.9) node[left] {$y$};
%f(t) 
\draw[thick] plot[variable=\x,domain=0:1,samples=50,smooth](\x,{f(\x)});
\draw[thin,gray,densely dashed] (1,0)--(1,{f(1)});
\draw[fill] (1,{f(1)}) circle (0.7pt); 
\draw[fill] (0,{f(0)}) circle (0.7pt);
\node[above] at(1,0) {$\pi$}; 
\node[below left] at(0,0) {$0$};
\visible<3->{
    \draw[thick] plot[variable=\x,domain=-1:0,samples=50,smooth](\x,{f(-\x)});
    \draw[thin,gray,densely dashed] (-1,0)--(-1,{f(1)});
    \node[above] at(-1,0) {$-\pi$};
    \draw[fill] (-1,{f(1)}) circle (0.7pt); 
    \draw[fill] (0,{f(0)}) circle (0.7pt);
}
\visible<4->{
    \foreach \k in {-3,-1,1}{  
    	\draw[thick] plot[variable=\x,domain=-1:0,samples=50,smooth](\x+\k+1,{f(-\x)});
    	\draw[fill] (\k,{f(1)}) circle (0.7pt); 
    	\draw[fill] (\k+1,{f(0)}) circle (0.7pt);
    	\draw[thick] plot[variable=\x,domain=0:1,samples=50,smooth](\x+\k+1,{f(\x)});
    	\draw[fill] (\k+2,{f(1)}) circle (0.7pt); 
    	\draw (\k+1,{f(0)}) circle (0.7pt);
    }
    \draw[thin,gray,densely dashed] (-3,{f(1)})--(-3,{f(1)});
    \draw[thin,gray,densely dashed] (3,{f(1)})--(3,{f(1)});
    \draw[thin,gray,densely dashed] (-1,0)--(-1,{f(1)});
    \draw[thin,gray,densely dashed] (1,0)--(1,{f(1)});
    \foreach \k in {-3,3} { 
    	\node[above] at(\k,0) {$\k\pi$}; 
    }
}
\end{tikzpicture}
\end{frame}

\begin{frame}
\begin{example}
将下面函数分别展开成正弦级数和余弦级数
\[ f(x)=\left\{\begin{array}{ll}
\cos x, & \quad 0 \leq x < \pi/2,\\
0, & \quad \pi/2 \leq x \leq \pi.
\end{array}\right.. \]
\end{example}
\end{frame}

\begin{sframe}
\frametitle{积化和差公式}
\begin{align*}
\sin x\cos y &= \phantom{-{}}\frac12\Big(\sin(x+y) + \sin(x-y)\Big) \\
\cos x\sin y &= \phantom{-{}}\frac12\Big(\sin(x+y) - \sin(x-y)\Big) \\
\cos x\cos y &= \phantom{-{}}\frac12\Big(\cos(x+y) + \cos(x-y)\Big) \\
\sin x\sin y &= -\frac12\Big(\cos(x+y) - \cos(x-y)\Big)
\end{align*}
\end{sframe}

\mybookmark{复习与提高}

\begin{frame}
\frametitle{复习与提高}
\vspace{-1em}%
\begin{choice}%[1999年数学一修改]
将$f(x)=\begin{cases}
  x, & 0\le x\le \frac{\pi}{2} \\
  2-2x, & \frac{\pi}{2}<x<\pi
\end{cases}$展开成傅里叶级数
$S(x)=\dfrac{a_0}{2}+\sum\limits_{n=1}^{\infty}a_n\cos nx$($-\infty<x<\infty$),
其中$a_n=\frac{2}{\pi}\int_0^{\pi}f(x)\cos nx\dx$($n=0,1,2,\cdots$),\vspace{0.5em}%
则$S\big(\!-\frac{5\pi}2\big)$等于\dotfill(\select{C})
\begin{choicequar}
  \item $\dfrac{\pi}{2}$  ~
  \item $-\dfrac{\pi}{2}$ ~
  \item $\dfrac{3\pi}{4}$ ~
  \item $-\dfrac{3\pi}{4}$ ~
\end{choicequar}
\end{choice}
\stext{\pause
\begin{solution}\smark
偶延拓,$S\big(\!-\frac{5\pi}2\big)=S\big(\!-\frac{\pi}2\big)=S\big(\frac{\pi}2\big)$.
\end{solution}}
\end{frame}

\section{一般周期函数的傅里叶级数}

\begin{frame}
假设$f(x)$是定义在$\mathbb{R}$上周期函数,周期为$T=2l$,其傅里叶级数为: 
\[
\frac{a_{0}}{2}+\sum_{n=1}^{\infty}\bigg(a_{n}\cos\frac{n\pi x}{l}+b_{n}\sin\frac{n\pi x}{l}\bigg)
\]
其中
\begin{align*}
 & a_{n}=\frac{1}{l}\int_{-l}^{l}f(x)\cos\frac{n\pi x}{l}\dx\quad(n=0,1,2,3,\cdots)\\
 & b_{n}=\frac{1}{l}\int_{-l}^{l}f(x)\sin\frac{n\pi x}{l}\dx\quad(n=0,1,2,3,\cdots)
\end{align*}
\end{frame}

\begin{frame}
\begin{example}
设$f(x)$是周期为$4$的周期函数,它在$[-2,2]$上的表达式为
\[ f(x)=\left\{\begin{array}{ll}
0, & \quad -1 \leq x < 0,\\
h, & \quad 0 \leq x \leq 2.
\end{array}\right.. \]
将$f(x)$展开成傅里叶级数.
\end{example}
\end{frame}

\mybookmark{复习与提高}

\begin{frame}
\frametitle{复习与提高}
\begin{puzzle}
将$f(x)=2+|x|$($-1\le x\le 1$)展开成以$2$为周期的傅里叶级数.
\end{puzzle}
\stext{\pause\begin{solution}\smark
$b_n=0$,$a_0=5$,$a_n=\dfrac{2}{n^2\pi^2}\Big((-1)^n-1\Big)$.
\end{solution}}
\end{frame}

\fi % <<<<<<<<<<<<<<<<<<<<<<<<<<<<<<<<<<<<<<<<<<<<<<<<<<<<<<<<<<<<<<<<<<<<<<<<<<

\end{document}
